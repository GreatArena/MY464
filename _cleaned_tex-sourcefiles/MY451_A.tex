\backmatter
%\addtocontents{toc}{\protect\newpage}
%\renewcommand{\appendixname}{ }
%\appendix

%\setlength{\parsep}{5cm}

\renewcommand{\chaptername}{}
\renewcommand{\thechapter}{}
\renewcommand{\thesection}{}
\renewcommand{\thesubsection}{}
\renewcommand{\chaptermark}[1]{\markboth{\MakeUppercase{#1}}{}}
\renewcommand{\sectionmark}[1]{\markright{\MakeUppercase{#1}}}
\renewcommand{\thetable}{A.\arabic{table}}
%\setcounter{chapter}{2}

\chapter{Computer classes}
\label{c_class0}

\setcounter{section}{-1}

Some general instructions on computing and the SPSS package are given
first below. It makes most sense to read these together with the
instructions for individual computer classes, which begin on page
\pageref{p_class1}.

\section{General instructions}

\subsubsection{Using the networked computers at LSE}


\begin{itemize}
\item
To access IT facilities at LSE you need an IT account with its
\textbf{Username} and \textbf{Password}. Please see
\texttt{
http://www.lse.ac.uk/intranet/LSEServices/IMT/}\\\texttt{guides/accounts/activateAccount.aspx}
for instructions on how to activate your account. In case of any
problems, please ask for assistance at the IT help desk (Library 1st
floor).
\item
Various introductory \textbf{documents}
can be accessed through the IMT services web pages at
\texttt{http://www.lse.ac.uk/intranet/LSEServices/IMT/home.aspx}.
\item
\textbf{Logging in} to use Windows:
When you arrive at a networked computer, wait for Windows to start up
(if the machine is not already on). Type in \textbf{CTRL + ALT + Delete} and
the \textbf{Enter Network Password} screen will appear. Type in your
username and your password and press \textbf{Enter} or click on the
\textbf{OK} button. This will log you on to the computer.
\end{itemize}


\subsubsection{Data downloading}

The instructions for each class will give the name of a file or files
which will be used for that exercise.  In order to do the class,
you will need to download the file to your H: space (i.e.\ your
personal file storage space on the LSE network, shown as disk drive H:
on a networked computer once you have logged on). You can download all
the data files for the course, as well as other course-related material,
from the web-based \textbf{Moodle} system. See instructions on page
\pageref{sss_moodle} for how to register for MY451 on Moodle.

%Alternatively, you can download the files from the Methodology
%Institute \textbf{Webpage}:
%\texttt{www2.lse.ac.uk/methodologyInstitute/data/dataforMi451.aspx}



%\begin{enumerate}
%\item
%Double click \textbf{Microsoft Outlook} icon on the screen in Windows (if you
%cannot find the icon, click on \textbf{Start}, then on
%\textbf{Programs}, followed by
%\textbf{Microsoft Outlook}).
%\item
%In Outlook click \textbf{View} and \textbf{Folder List} if the folder
%list is not shown. Scroll down until you find \textbf{Public Folders}.
%Click \textbf{Public Folders}, then \textbf{All Public Folders}, then
%\textbf{Departments}, then \textbf{Methodology Institute}.
%\item
%Click the folder for \textbf{MI451}. Then click on \textbf{Class and
%Homework}. Highlight the first file in the list.
%\item
%Click on \textbf{File} and \textbf{Save As}.
%\item
%Save a copy of the file on to your H: space by choosing H: (or some
%other folder in which you want to save the files) in \textbf{Save in:}
%and clicking \textbf{Save}.
%\item
%Repeat 3. to 5. for all files in the MI451 folder until you have copied
%them all onto your H: space.
%\end{enumerate}
%Alternatively, you can download the files from the Methodology
%Institute \textbf{Webpage}:
%\texttt{www.lse.ac.uk/collections/methodologyInstitute/data/dataforMi451.htm}

%\subsubsection{Moodle}
%
%The web-based \textbf{Moodle} system also provides access to the data
%files and other course-related material. See instructions on page
%\pageref{sss_moodle} for how to register for Mi451 on Moodle.

\section{Introduction to SPSS}

%\emph{Note on the name}: In Version 18 of the software, SPSS is
%officially called \textbf{PASW}, short for ``Predictive Analysis
%Software''.
%This is the name you may see in the Windows menus and within the
%programme itself. However, we will continue to call it SPSS in these
%instructions.  The most recent versions of the
%programme are ``SPSS'' again, so the brief attempt at
%rebranding can be ignored.

\subsubsection{General information and documentation}

SPSS (formerly Statistical Package for the Social Sciences) is a widely
used general-purpose statistical software package. It will be used for
all the computer classes on this course. The current version on the  LSE
network is SPSS 21. This section gives some general information on the
structure and use of SPSS. The discussion is brief and not meant to
be comprehensive. The instructions given here and in the descriptions of
individual computer classes below will be sufficient for the purposes of
this course. If, however, you wish to find out more about SPSS, more information
and examples can be found in
the SPSS help files and tutorials found under the \textbf{Help} menu of
the program, and in introductory
guide books such as\label{p_spssbooks}

Field, A.\ (2013). \emph{Discovering Statistics using IBM SPSS
Statistics} (4th ed). Sage.\\
Kinnear, P.\ R.\ and Gray, C.\ D.\ (2012). \emph{SPSS 19 Made Simple}.
Psychology Press.\\
Pallant, J. (2013). \emph{SPSS Survival Manual} (5th ed). Open
University Press.

These are given here purely as examples (there are many others) and not
as recommendations. We have not reviewed any of these books in detail
and so cannot make any comparisons between them.

\subsubsection{Starting SPSS}

To start SPSS, double-click on the SPSS icon on the Windows desktop.
Alternatively, click on the \textbf{Start} button at the
bottom left corner, and select \textbf{All Programs}, then
\textbf{Specialist and teaching software}, \textbf{Statistics}, \textbf{SPSS}, and finally
\textbf{SPSS 21}
(or some obvious
variant of these, in case the exact wording on your desktop is slightly
different).

An initial screen for opening data files appears. Click on
\textbf{Cancel} to get rid of this and to enter the data editor (which
will be discussed further below).


\subsubsection{Exiting from SPSS}

Select \textbf{Exit} from the \textbf{File} menu or click on the X at
the upper right corner of the SPSS data editor window. You may then be
prompted to save the information in the open windows; in particular, you
should save the contents of the data editor in a file (see below) if you
have made any changes to it.

\subsubsection{SPSS windows}

There are several different types of windows in SPSS. The two most
important are
\begin{itemize}
\item
\textbf{Data editor}: A data set is displayed in the Data Editor window.
Several of these can be open at a time.
The data editor which you have selected (clicked on) most recently defines the active data set,
and the procedures you request from the menus are applied to this data
set until you select a different active data set.
The data editor window has two parts,
accessed by clicking on the two tabs at the bottom of the window:
\begin{itemize}
\item
\textbf{Data view}, which shows the data matrix in the spreadsheet-like form discussed in
Section \ref{ss_intro_def_subj}, with units in the rows and variables in
the columns.
\item
\textbf{Variable view}, which shows information about the variables.
\end{itemize}
Working with the data editor will be practised in the first computer
class. The contents of the data editor, i.e.\
the data matrix and associated information, can be saved in an SPSS data
file. Such files have names with the extension \textbf{.sav}.
\item
\textbf{Output viewer}: Output from statistical analyses carried out on
the data will appear here. The output can be printed directly from the
viewer or copied and pasted to other programs. The contents of the
viewer can also be saved in a file, with a name with the extension
\textbf{.spv} (since version 17; in previous versions
of SPSS the extension was
\textbf{.spo}).
\end{itemize}
There are also other windows, for example for editing SPSS graphs. They
will be discussed in the instructions to individual computer classes
where necessary.

\subsubsection{Menus}

SPSS has a menu-based interface, which can be used to access most of its
features for statistical analysis, manipulation of data, loading, saving
and printing files, and so on.
\begin{itemize}
\item
The procedures for statistical analysis are found under the
\textbf{Analyze} menu, which provides further drop-down menus for
choosing different methods.
\begin{itemize}
\item
Similarly, procedures for various
statistical graphics are found under \textbf{Graphs}.
We will be using procedures found under \textbf{Graphs / Legacy
Dialogs}. Here ``legacy'' means that these are the graphics menus which
were included also in previous versions of SPSS.
The current version also contains a second, new set of menus for the
same graphs, under \textbf{Graphs / Chart Builder}. We do not regard
these as an improvement in usability, so we will continue to use the old
menus. You are welcome to explore the cababilities of the ``Chart
Builder'' on your own.
\end{itemize}
\item
Eventually the menu choices lead to a \textbf{dialog box} with various
boxes and buttons for specifying details of the required analysis. Most
dialog boxes contain buttons which open new dialog boxes for further
options. The details of these choices for the methods covered on this
course are described in the instructions to individual computer classes.
\item
Almost all of the dialog boxes have options which are not needed for our
classes and never mentioned in the instructions. Some of these simply
modify the output, others request variants of the statistical methods
which will not be used in these
classes. All such options have default values which can be left
untouched here. You are, however, welcome to
experiment with these additional choices to see what they do. Further
information on them can be accessed through the \textbf{Help} button in
each dialog box.
\end{itemize}


\subsubsection{Notational conventions for the instructions}

Because analyses in SPSS are carried out by making choices from the
menus, the instructions for the computer classes need to describe these
choices somehow. To reduce the length and tedium of the instructions, we
will throughout present them in a particular format explained below.
Because this information is rather abstract if read in isolation, it is best to
go through it while carrying out specific instructions for the first few
computer classes.
\begin{itemize}
\item
The appropriate menu choices for obtaining the dialog box for the
required analysis are first given in bold, for example as
follows:

\textbf{Analyze/Descriptive statistics/Frequencies}

This is short for ``Click on the menu item \textbf{Analyze} at the top
of the window; from the drop-down menu, select \textbf{Descriptive
statistics} and then click on \textbf{Frequencies}.'' This particular
choice opens a dialog box for constructing various descriptive
statistics and graphs (as discussed in Chapter \ref{c_descr1}).

Unless otherwise mentioned, subsequent instructions then refer to
choices in the most recently opened dialog box, without repeating the
full path to it.
\item
For all of the statistical analyses, we need first to specify which
variables the analyses should be applied to. This is done by entering
the names of those variables in appropriate boxes in the dialog boxes.
For example, the dialog box opened above has a box labelled
\textbf{Variable(s)} for this purpose. The dialog box also includes a
separate box containing a list of all the variables in the data set. The
required variables are selected from this list and moved to the choice
boxes (and back again, when choices are changed) by clicking on an arrow
button between the boxes. For example, suppose that a data set contains
a grouped age variable called \emph{AGEGROUP}, for which we want to construct a
frequency table. The class instructions may then state in words ``Place
\emph{AGEGROUP} in the \textbf{Variable(s)} box'', or sometimes just

\textbf{Variable(s)/}\emph{AGEGROUP}

both of which are short for ``In the dialog box opened above, click on
the name \emph{AGEGROUP} in the list of variables, and then click on the
arrow button to move the name into the \textbf{Variable(s)} box''.
Sometimes we may also use a generic instruction of the form

\textbf{Variable(s)/}\emph{$<$Variables$>$}

where \emph{$<$Variables$>$} indicates that this is where we would put
the name of any variables for which we want to obtain a frequency
table. Note that here and in many other procedures, it is possible to
select several variables at once. For the Frequencies procedure used as
an example here, this simply means that a separate frequency table is
constructed for each selected variable.
\item
Other choices in a dialog box determine details of the analysis
and its output. In most cases the selection is made from a fixed
list of possibilities provided by SPSS, by clicking on the appropriate
box or button. In the instructions, the choice is indicated by listing a
path to it, for example as

\textbf{Charts/Chart Type/Bar charts}

in the above example (this requests the so-called bar chart). The items on such a list are
labels for various items in the dialog boxes. For example, here
\textbf{Charts} is a button which opens a new subsidiary dialog box,
\textbf{Chart Type} is the title of a list of options in this new dialog
box, and \textbf{Bar charts} is the choice we want to select.
In other
words, the above instruction is short for ``In the dialog box opened
above, click on the button \textbf{Charts} to open a new dialog box.
Under \textbf{Chart type}, select \textbf{Bar charts}
by clicking on a button next to it.''
\item
Some choices need to be made by typing in some information
rather than selecting from a list of options.
Specific instructions for this will be given when needed.
\item
After choices are made in subsidiary dialog boxes, we return to the main
dialog box by clicking on \textbf{Continue}. Once all the required
choices have been made, the analysis is executed by clicking on
\textbf{OK} in the main dialog box. This should be reasonably obvious,
so we will omit explicit instructions to do so.
\end{itemize}

A useful feature of SPSS is the \textbf{dialog recall} button, which is
typically sixth from the left in the top row of buttons in the
Output viewer window; the button shows a rectangle with a green arrow
pointing down from it. Clicking on this gives a menu of recently used procedures, and
choosing one of these brings up the relevant dialog box, with the
previously used choices selected. This is useful when you want to rerun
a procedure, e.g.\ to try different choices for its options. It is
usually quicker to reopen a dialog box using the dialog recall button
than through the menus.


\subsubsection{SPSS session options}

Various options for controlling the format of SPSS output and other
features can be found under \textbf{Edit/Options}. For example,
an often useful choice is
\textbf{General/Variable Lists/Display names}\label{p_listoptions}. This
instructs SPSS to display the names of
variables in the variable lists of all procedures, instead of the
(typically much longer) descriptive labels of the variables. In large
data sets this may make it easier to find the right variables from the
list. This may be further helped by
selecting \textbf{General/Variable Lists/Alphabetical}, which causes the
names to be listed in an alphabetical order rather than the
order in which the variables are included in the data set.


\subsubsection{Printing from SPSS}


All the computers in the public rooms are connected to one of the
laser printers. When you print a document or a part of it, you need to
have credit on your printing account.
See
\small{\texttt{http://www.lse.ac.uk/intranet/LSEServices/IMT/guides/printing.aspx}}
\normalsize
for further information.

\begin{itemize}
\item
You can print your results from the Output Viewer either by selecting
\textbf{File/Print} or by clicking on Print on the toolbar (the button
with a little picture of a printer). Please note that SPSS output is
often quite long, so this may result in much more printout than you
really want.
\item
Alternatively, in the Output Viewer, select the objects to be printed,
select \textbf{Edit / Copy}, open a Word or Excel document
and \textbf{Paste}. You can make any changes or corrections in this
document before printing it. This method gives you more control over
what gets printed than printing directly from SPSS.
\item
At the printer terminal, type in your username and password. The files
sent for printing are then listed.  Select the appropriate file number
and follow the instructions given by the computer.
\end{itemize}

\subsubsection{SPSS control language}

Early versions of SPSS had no menu-based interface. Instead, commands
were executed by specifying them in SPSS command language. This language
is still there, underlying the menus, and each choice of commands and
options from the menus can also be specified in the control language. We
will not use this approach on this course, so you can ignore this
section if you wish. However, there are some very good reasons why you
might want to learn about the control language if you need to work with
SPSS for, say, analyses for your thesis or dissertation:
\begin{itemize}
\item
Because the control language commands can be saved in a file, they
preserve a record of how an analysis was done. This may be important for
checking that there were no errors, and for rerunning the analyses later
if needed.
\item
For repetitive analyses, modifying and rerunning commands in the control
language is quicker and less tedious than using the menus repeatedly.
\item
Some advanced SPSS procedures are not included in the menus, and
can only be accessed through the control language.
\end{itemize}
The main cost of using the control language is learning its syntax. This
is initially much harder than using the menus, but becomes easier with
experience. The easiest way to begin learning the syntax is to request
SPSS to print out the commands corresponding to choices made from the
menus. Two easy ways of doing so are
\begin{itemize}
\item
Selecting the session option (i.e.\ under \textbf{Edit/Options})
\textbf{Viewer/Display commands in the log}. This causes the commands
corresponding to the menu choices to be displayed in the output window.
\item
Clicking on the \textbf{Paste} button in a dialog box (instead of
\textbf{OK}) after selecting an analysis. This opens a \emph{Syntax
window} where the corresponding commands are now displayed. The commands
in a syntax window can be edited and executed, and also saved in a file
(with the extension \textbf{.sps}) for future use.
\end{itemize}

\newpage
%$\; $
%\newpage
\section[Week 2: Descriptive statistics 1]{WEEK 2 class: Descriptive statistics for categorical data, and entering data}
\label{p_class1}

\subsubsection{Data set}

The data file \textbf{ESS5\_sample.sav} will be used today.
It contains a simplified sample of data
from UK respondents in the 2010 European Social Survey (Round 5).
The questions in the survey that you see here were designed
By Dr Jonathan Jackson and his team as part of a module investigating
public trust in the criminal justice system. Further information about the
study can be found at\\
\texttt{www.lse.ac.uk/methodology/whosWho/Jackson/jackson\_ESS.aspx}
\footnote{ESS Round 5: European Social Survey Round 5 Data (2010). Data file edition 2.0. Norwegian Social Science Data Services, Norway � Data Archive and distributor of ESS data. The full data can be
obtained from \texttt{http://ess.nsd.uib.no/ess/round5/}.}

The main purpose of today's class is to introduce you to the layout of SPSS and to show you
how to produce some basic tables and graphs for categorical variables.
Additionally, we provide instructions on how to enter data into a new SPSS
data file, using the Data Editor. This exercise is not strictly needed for the course, but
we include it for two purposes. Firstly, students often find this a helpful
way of learning how the software works. Secondly, this exercise may be a useful introduction
for students who go on to collect or collate data for their own empirical research.

%\footnote{
%There is
%also a third approach, not considered today,
%which is to read in a data file stored initially
%in some format, such as a spreadsheet file or a simple text file.}.

%This is done by choosing \textbf{File / Open / Data}, selecting the file
%from under \textbf{Files of type} of the required kind, and following
%subsequent instructions. We will not discuss this possibility today. If
%needed, it is usually reasonably straightforward to learn, at least after some
%trial and error.

\subsubsection{Classwork}

\subsubsection{Part 1: The layout of an SPSS data file}

\begin{enumerate}
\item
\textbf{Opening an SPSS data file}: this is done from
\textbf{File/Open/Data}, selecting the required file from whichever folder it is
saved in in the usual Windows way. Do this to open ESS5\_sample.sav.
\item
\textbf{Information in the Variable View window.} The data file is now
displayed in the Data Editor. Its Data View window shows the data as a
spreadsheet (i.e.\ a data matrix). We will first consider the
information in the Variable View window, accessed by clicking on the
Variable View tab at the bottom left corner of the window. The columns
of this window show various pieces of information about the variables.
Take a little while familiarising yourself with them. The most important
of the columns in Variable View are
\begin{itemize}
\item
\textbf{Name} of the variable in the SPSS data file. The names in this
column (also shown as the column headings in Data View) will be
used to refer to specific variables in all of the instructions for these
computer classes.
\item
\textbf{Type} of the variable. Here most of the variables are
\emph{Numeric}, i.e.\ numbers, and a few are
\emph{String}, which means text.
Clicking on the entry for a variable in
this column and then on the button (with three dots on it) revealed by
this shows a list of other possibilities.
\item
\textbf{Width} and \textbf{Decimals} control the total number of digits
and the number of decimal places displayed in Data View. Clicking on
an entry in these columns reveals buttons which can be used to increase
or decrease these values.
Here all but two of the numeric variables are coded as whole numbers, so
Decimals has been set to 0 for them.
\item
\textbf{Label} is used to enter a longer description of the variable.
Double-clicking on an entry allows you to edit the text.
\item
\textbf{Values} shows labels for individual values of a variable. This is
mostly relevant for categorical variables, such as most of the ones
in these data. Such variables are coded in the data set as numbers, and
the Values entry maintains a record of the meanings of the categories
the numbers correspond to. You can see examples of this by clicking on
some of the entries in the Values column and then on the resulting
button. The value labels can also be displayed for each observation in
Data View by selecting \textbf{View/Value Labels} in that
window.
\item
\label{p_missingcodes}
\textbf{Missing} specifies \emph{missing data codes}, i.e.\ values which
are not actual measurements but indicators that an observation should be
treated as missing. There may be several such codes. For example,
variables in these data often have separate missing data codes for cases
where a
respondent was never asked a question (``Not applicable'', often
abbreviated NAP), replied
``Don't know'' (DK) or otherwise failed to provide an answer (``Refusal'' or
``No answer''; NA); the explanations of these values are found in the
Values column.
An alternative to using missing data codes
(so-called \emph{User missing} values)
is to enter no value (a \emph{System missing} value)
for an observation in the data matrix.
This is displayed as a full stop (.) in Data View. There are no such
values in these data.
\item
\textbf{Measure} indicates the measurement level of a variable, as
\emph{Nominal}, \emph{Ordinal} or \emph{Scale} (meaning
interval). This is mostly for the user's information, as SPSS
makes little use of this specification.
\end{itemize}
\item
Any changes made to the data file are preserved by saving it again from
\textbf{File/Save} (or by clicking on the Save File button of the
toolbar, which the one with the picture of a diskette). You will also be
prompted to do so when exiting SPSS or when trying to open a new data
file. Today you should not save any changes you may have made to
ESS5\_sample.sav, so click \textbf{No} if prompted to do so below.
\end{enumerate}

%\newpage
\subsubsection{Part 2: Descriptive statistics for categorical variables}

Most of the statistics required for this class are found in SPSS
under \textbf{Analyze/Descriptive Statistics/Frequencies} as follows:
\begin{itemize}
\item
Names of the variables for which the statistics are requested are
placed in
the \textbf{Variable(s)} box.
To make it easy to find variables in the list box on the left, you may
find it convenient to change the
way the variables are displayed in the list; see page
\pageref{p_listoptions} for instructions.
\item
Tables of frequencies: select \textbf{Display frequency tables}
\item
Bar charts: \textbf{Charts/Chart Type/Bar charts}. Note that under \textbf{Chart Values}
you can choose between frequencies or percentage labels on the vertical axis.
\item
Pie charts: \textbf{Charts/Chart Type/Pie charts}
\end{itemize}
In addition, we will construct some two-way tables or cross-tabulations,
by selecting \textbf{Analyze/Descriptive Statistics/Crosstabs}. In the
dialog box that opens, request a contingency
table between two variables by entering
\begin{itemize}
\item
The name of the row
variable into the \textbf{Row(s)} box, and
\item
The name of the column
variable into the \textbf{Column(s)} box.
\item
\textbf{Cells/Percentages} for percentages within the table:
\textbf{Row} gives percentages within each row (i.e.\ frequencies
divided by row totals), \textbf{Column}
percentages within columns, and
\textbf{Total} percentages out of the total sample size.
\end{itemize}

The labels in the SPSS output should be self-explanatory. Note that in
this and all subsequent classes, the output may also include some
entries corresponding to methods and statistics not discussed on this
course. They can be ignored here.

\begin{enumerate}
\item
The first variable in the data set, GOODJOB, asks respondents whether they generally
feel that the police are doing a good job in their country. There are three response categories for
this item: ``a good job'', ``neither a good job nor a bad job'', or ``a
bad job''.
Obtain a frequency table and bar chart to investigate the distribution of responses to this
question.

Check that you understand how to interpret the output you obtain. In particular,
make sure that you understand the information displayed in each of the columns in the main
table, and that you can see the connection between the information in the table and
the information represented in the bar chart.

\item
The last variable in the set, AGE\_GRP, records in which of the following age groups each
respondent falls: up to 29 years of age, 30-49, or 50+ years.
Let us consider the association between age group and opinions of the police. Obtain a
two-way contingency table of GOODJOB by AGE\_GRP. To make interpretation easier, request
 percentages within each of the age groups. If you use AGE\_GRP as the row variable, then include
 row percentages in the output.

Interpret the resulting table. Are opinions of the police distributed differently among the three
different age groups? Does there appear to be an association between age group and attitude?

\item
If you have time after completing the data entry exercise (below), you may wish to return to
this data set and explore frequencies and contingency tables for some of the other variables in
the set.
\end{enumerate}

\newpage
\subsubsection{Part 3: Entering data directly into Data Editor}

This procedure may be
useful to know if the data you are analysing are not in any electronic
form at the beginning of the analysis,
for example if you start with a pile of filled-in questionnaires from a
survey. For practice, we will enter the following small,
artificial data set:
\begin{center}
Sex: Man; Age: 45; Weight: 11 st 3 lbs\\
Sex: Woman; Age: 68; Weight: 8 st 2 lbs\\
Sex: Woman; Age: 17; Weight: 9 st 6 lbs\\
Sex: Man; Age: 28; Weight: 13 st 8 lbs\\
Sex: Woman; Age: 16; Weight: 7 st 8lbs
\end{center}
\begin{enumerate}
\item
Select \textbf{File/New/Data} to clear the Data Editor. Go to Variable
View and enter into the first four rows of the Name column names for the
variables, for example \emph{sex}, \emph{age}, \emph{wstones} and
\emph{wpounds}.
\item
Switch to Data View and type the data above
into the appropriate columns, one unit (respondent) per
row. Note that the person's weight is entered into two columns, one for
stones and one for pounds. Enter sex using numerical codes, e.g.\ 1 for
women and 2 for men.
\item
Save the file as a new SPSS data file (\textbf{File/Save as}), giving it
a name of your choice.
You should also resave the file (from \textbf{File/Save} or by
clicking the File Save button)
after each of the changes made below.
\item
Practise modifying the information in Variable View by adding
the following information for the sex variable:
\begin{itemize}
\item
Enter the label \emph{Sex of the respondent} into the Label column.
\item
Click on the Values cell and then on the resulting button to open a
dialog box for entering value labels.
Enter \textbf{Value:}
\emph{1}; \textbf{Value Label}: \emph{Woman}; \textbf{Add}. Repeat for
men, and click \textbf{OK} when finished.
\end{itemize}
%\end{enumerate}
\item
\textbf{Transforming variables}: It is often necessary to
derive new variables from existing ones.
We will practise the two most common examples of this:
\begin{enumerate}
\item
\textbf{Creating a grouped variable}:
Suppose, for example, that we want to define a grouped age variable
with three categories: less than 25 years, 25--54 and 55 or over. This is done
as follows:
\begin{itemize}
\item
Select \textbf{Transform/Recode into Different
Variables}. This opens a dialog box which is used to define the rule for
how values of the existing variable are to be grouped
into categories of the new one.
\item
Move the name of the age
variable to the \textbf{Input Variable --$>$ Output Variable} box.
\item
Under \textbf{Output Variable}, enter
the \textbf{Name} of the new variable, for example
\emph{agegroup}, and click \textbf{Change}.
\item
Click on \textbf{Old and New Values}. Enter \textbf{Old Value/Range:
Lowest through} \emph{24} and \textbf{New Value/Value:} \emph{1}, and
click \textbf{Add}.
\item
Repeat for the other two categories, selecting
\textbf{Range:} \emph{25} \textbf{through} \textbf{54} and
\textbf{Range:} \emph{55} \textbf{through highest}
for \textbf{Old value}, and \emph{2} and \emph{3} respectively
for \textbf{New value}.
\item
You should now see the correct grouping instructions in the \textbf{Old
--$\mathbf{>}$ New} box. Click \textbf{Continue} and \textbf{OK} to
create the new variable.
\item
Check the new variable in Data View. At this stage
you should normally enter in Variable View the value labels of the age
groups.
\end{itemize}
\item
\textbf{Calculations on variables}: Some new variables are obtained
through mathematical calculations on existing ones. For example, suppose
we want to include weight in kilograms as well as stones and pounds.
Using the information that one stone is equal to 6.35~kg and one pound
is about 0.45~kg, the transformation is carried out as follows:
\begin{itemize}
\item
Select \textbf{Transform/Compute Variable}. This opens a dialog box which is
used to define the rule for calculating the values of the new variable.
\item
Enter \textbf{Target variable:} \emph{weightkg} (for example; this is
the name of the new variable) and
\textbf{Numeric Expression:} \emph{6.35 * wstones + 0.45 * wpounds}; for
the latter, you can either type in the formula or use the variable list and
calculator buttons in a fairly obvious way.
\end{itemize}
\end{enumerate}
\end{enumerate}


\subsubsection{WEEK 2 HOMEWORK}

The homework exercise for this week is to complete the multiple choice quiz
which you can find in the Moodle resource
for MY451. Answers to the questions are also included there, including feedback on
why the incorrect answer are incorrect. The first part of the quiz asks for answers
to the class exercise, and the second part asks you to identify the level of measurement
of some different variables.

\clearpage
%\newpage
%$\, $
%\newpage
\section[Week 3: Descriptive statistics 2]{WEEK 3 class:\\Descriptive statistics for continuous variables}

%\subsubsection{Data set}
\textbf{Data set:}
The data file used today is \textbf{london-borough-profiles.sav}. It contains
a selection of data on the 33 London boroughs obtained from the \emph{London Datastore},
which publishes a range of statistical data about the city, collated by the Greater London Authority's
\emph{GLA Intelligence Unit}
\footnote{The data were
obtained from
{\scriptsize\texttt{http://data.london.gov.uk/datastore/package/london-borough-profiles}}.

\noindent If you download the ``Profiles in Excel'' workbook, you will find that one of the pages contains
a map of the boroughs, and a tool for visualising the data on that map. A regular map of the boroughs
can be found at for example at

{\scriptsize\texttt{http://www.londoncouncils.gov.uk/londonfacts/londonlocalgovernment/londonmapandlinks/default.htm}}.}.


\subsubsection{Descriptive statistics in SPSS}


This week you will produce and examine descriptive statistics for a number of
individual variables.
As for last week, almost all of the statistics
required for this class can be obtained in SPSS
under \textbf{Analyze/Descriptive Statistics/Frequencies}. Note that you will probably
not find the tables of frequencies very useful, because continuous variables can take
so many different values. So for this class, uncheck the \textbf{Display frequency tables}
option in the dialog box.
\begin{itemize}
\item
Measures of central tendency: \textbf{Mean}, \textbf{Median} and
\textbf{Mode} under \textbf{Statistics / Central Tendency}
\item
Measures of variation: \textbf{Range}, \textbf{Std.\ deviation} and
\textbf{Variance}
under \textbf{Statistics/Dispersion}. For the Interquartile
range, select \textbf{Statistics/ Percentile values/Quartiles}
and calculate by hand the difference between the third and first quartiles
given (as Percentiles 75 and 25 respectively) in the output.
\item
Histograms: \textbf{Charts/Chart Type/Histograms}
\end{itemize}
Two charts needed today
are not found under the Frequencies procedure:
\begin{itemize}
\item
\textbf{Stem and leaf plots}, which
are obtained from \textbf{Analyze/Descriptive Statistics/Explore} by
entering variable(s) under \textbf{Dependent list} and selecting
\textbf{Display/Plots} and \textbf{Plots/Descriptive/Stem-and-leaf}. You can place
 more than one variable under the \textbf{Dependent list} in order to compare variables.
\item
\textbf{Box plots} are also automatically generated through this dialog box,
regardless of whether you want to see them! So this is the simplest way
to produce them.
\end{itemize}
Most of these statistics and charts can be obtained in other ways as well, for example from
\textbf{Analyze/ Descriptive Statistics/Descriptives} or
\textbf{Graphs/Legacy Dialogs/Histogram}, or \textbf{Graphs/Legacy Dialogs/Boxplot},
but we will not use these alternatives today. Feel free to investigate them in your own time if
you wish.

The labels in the SPSS output should be self-explanatory. Note that in
this and all subsequent classes, the output may also include some
entries corresponding to methods and statistics not discussed on this
course. They can be ignored here.

\subsubsection{Classwork}

\begin{enumerate}
\item
The variable \emph{YOUTH\_DEPRIVATION} records for each borough the percentage
of children who live in out-of-work families. This is an indicator
of deprivation, with higher values indicating a worse situation for each borough. Investigate
the distribution of this variable across London boroughs by obtaining its mean, median,
minimum and maximum, quartiles and standard deviation, and a histogram. Obtain also a stem and leaf
plot and a box plot.
Note that double-clicking on a histogram (or any other SPSS graph) opens it in a
new window, where the graph can be further edited by changing titles,
colours etc. The graph can also be exported from SPSS into other software.
Check that you understand how to find the measures of central tendency and dispersion from the
output. Does the distribution of YOUTH\_DEPRIVATION appear to be symmetrically distributed or
skewed?
\item
Consider now the variable \emph{CRIME}, which records the numbers of
reported crimes for every 1000 inhabitants, over the years 2011-12. Obtain
some summary descriptive statistics, a histogram and a box plot for this variable.
Is the distribution of the variable symmetric or skewed to the
left or right?
\emph{CRIME} is one of many variables in this data set which have
outliers, i.e.\ boroughs with unusually large or small values of the variable.
Normally statistical analysis focuses on the
whole data rather than individual observations, but
the identities of individual outliers are often also of interest. The outliers
can be seen most easily in the box plots, where SPSS labels them with their
case numbers, so that you can identify them easily in the data set. For example, 1 would indicate
the 1st case in the data set. If you click on to the Data View you can see that
this 1st case is the City of London. Which borough is the outlier for CRIME?
\end{enumerate}

\subsubsection{HOMEWORK}
For the questions below, select the relevant SPSS output to include in
your homework and write brief answers to the specific questions.
Remember SPSS produces some outputs that you do not need. Feel free to
transcribe tables or modify charts if you wish to improve their
presentation.

%questions?

\begin{enumerate}
\item
The variable \emph{VOTING} records voter turnout in a borough, specifically
the percentage of eligible voters who voted in the local elections in 2010.
Obtain descriptive statistics, a histogram and a box plot for this variable. What is the
range of the variable, and what is its inter-quartile range? Are there any outliers?
Is the distribution of voter turnout symmetrical or skewed? How you can you tell?
\item
In the data set employment rates are given overall, but also separately
for males and females. The employment rate is the percentage of working age population
who are in employment. Compare and contrast male and female employment
rates across the boroughs, using the variables \emph{MALE\_EMPLOYMENT}
and \emph{FEMALE\_EMPLOYMENT}. Comment on the differences and/or similarities in
their descriptive statistics: minimum and maximum, mean, median and standard deviation. Obtain
histograms for these two variables. Are the distributions
of male employment and female employment symmetrical or skewed?
\end{enumerate}


\newpage
\section[Week 4: Two-way contingency tables]{WEEK 4 class: Two-way contingency tables}

\textbf{Data set}:
The data file used today is \textbf{GSS2010.sav}. It contains a
selection of variables on attitudes and demographic characteristics for
2044 respondents in the 2010 U.S.\ General Social Survey
(GSS)\footnote{The data can be obtained from
\texttt{http://www3.norc.org/gss+website/}, which gives further information on
the survey, including the full text of the questionnaires. }. The full data set
contains 790 variables. For convenience the version you are analysing today
contains just a selection of those items.

\subsubsection{Analysing two-way contingency tables in SPSS}

All of the analyses needed for this week's class are found
under \textbf{Analyze/Descriptive Statistics/Crosstabs}. We will be
obtaining contingency tables between two variables, as in Week 2 class,
with the following commands:
\begin{itemize}
\item
The name of the row
variable into the \textbf{Row(s)} box, and
\item
The name of the column
variable into the \textbf{Column(s)} box.
\item
\textbf{Cells/Percentages} for percentages within the table:
\textbf{Row} gives percentages within each row (i.e.\ frequencies
divided by row totals), \textbf{Column}
percentages within columns, and
\textbf{Total} percentages out of the total sample size.
\end{itemize}

%\\[0.7\baselineskip]
%\hspace*{2em} the name of the row
%variable into the \textbf{Row(s)} box, and\\
%\hspace*{2em} the name of the column
%variable into the \textbf{Column(s)} box.

The only additional output we will need today is obtained by
selecting
\begin{itemize}
\item
\textbf{Statistics/Chi-square} for the $\chi^{2}$ test of
independence
\item
(If you are interested in the $\gamma$ measure of
association for ordinal variables, outlined in the coursepack,
you may obtain it using textbf{Statistics/Ordinal/Gamma}. In the output the
$\gamma$ statistic is shown in the ``Symmetric measures'' table in the
``Value'' column for ``Gamma''. We will not use this measure today, but
feel free to ask if you are interested in it.)
\end{itemize}

%\\[.7\baselineskip]
%\hspace*{2em}\textbf{Statistics/Chi-square} for the $\chi^{2}$ test of
%independence,\\
%\hspace*{2em}\textbf{Statistics/Ordinal/Gamma} for the $\gamma$ measure of
%association, and\\
%\hspace*{2em}\textbf{Cells/Percentages} for percentages within the table:\\
%\hspace*{3em}\textbf{Row} gives percentages within each row (i.e.\ frequencies
%divided by row totals), \textbf{Column}\\
%\hspace*{3em}percentages within columns, and
%\textbf{Total} percentages out of the total sample size.

%By default, the list of variables is given in the order the variables
%appear in the data file, and using their long labels. It may be more
%convenient to change this to show only the variable names in
%alphabetical order, as explained on page \pageref{p_listoptions}.

\subsubsection{Classwork}

Suppose we want to use the GSS data to investigate whether in the U.S.\ population
sex and age are associated with attitudes towards women's roles. The
respondent's sex is
included in the data as the variable \emph{SEX}, and age as
\emph{AGEGROUP} in three groups: 18-34, 35-54, and 55 or over.
The three attitude variables we consider are
\begin{itemize}
\item
\emph{FEFAM}: Level of agreement with the following statement: ``It is much better for
everyone involved if the man is the achiever outside the home and the woman takes care
of the home and family''. Available response options are
Strongly agree, Agree, Disagree, and Strongly disagree.
\item
FEPOL: Level of agreement with the following statement: ``Most men are better suited
emotionally for politics than are most women''. Available response options are:
Agree and
Disagree.
\item
\emph{FEPRES}: Response to the following statement: ``If your party nominated a woman
for President, would you vote for her if she were qualified for the job?''
Available response options are  Yes and No.
\end{itemize}

%\newpage
\begin{enumerate}
\item
Consider first the association between sex and attitude towards
male and female work roles, by constructing a contingency table between
\emph{SEX} and \emph{FEFAM}. To make interpretation of the
results easier, include also appropriate percentages. Here it makes most
sense to treat sex as an explanatory variable for attitude, so we want
to examine percentages of attitudes within categories of male and female. If you use
\emph{SEX} as the row variable, this means including the Row
percentages in the output. Request also the $\chi^{2}$-test statistic.
%and the $\gamma$ statistic. The latter is appropriate here because both
%the age and attitude variables are ordinal.

In SPSS output, results for the $\chi^{2}$ test are given below the
two-way table itself in a table labelled ``Chi-Square Tests'', in the
row ``Pearson Chi-Square''. The test statistic itself is given
under ``Value'' and its $P$-value under ``Asymp.\ Sig.\ (2-sided)''.
%(If you request the
%$\gamma$ statistic it is shown in the ``Symmetric measures'' table in the
%``Value'' column for ``Gamma''.)

By considering the $\chi^{2}$ test statistic and its $P$-value, do you
think there is enough evidence to conclude that males and females
differ in their views on male and female work roles?
If there is, how would you describe the association?
\item
Consider now the association between age and attitude towards
male and female work roles, by constructing a table between
\emph{AGEGROUP} and \emph{FEFAM}. Interpret the results, and compare
them to your findings in Exercise 1.
\item
Examine differences between men and women in their
views about women's suitability for politics,
using a table between
\emph{SEX} and \emph{FEPOL}. Interpret the results. (Note: ignore the
last two columns of the $\chi^{2}$ test output, labelled
`Exact\ Sig.\ (2-sided)'' and `Exact\ Sig.\ (1-sided)'', and use the
result under ``Asymp.\ Sig.\ (2-sided)'' as in the other tables.)
%(Note that the $\gamma$ statistic is here still appropriate
%but less useful than before bexause the sex variable
%has only two categories.)
\end{enumerate}


\subsubsection{HOMEWORK}

\begin{enumerate}
\item
What is the null hypothesis for the $\chi^{2}$ test that you carried out
in analysis 2 in the class, for the table of \emph{AGEGROUP} by \emph{FEFAM}?
\item
State the $\chi^{2}$ test statistic, degrees of freedom and $P$-value for this table, and
interpret these results.
\item
Interpret the table of percentages to describe the nature of the association
between \emph{AGEGROUP} and \emph{FEFAM}.
\item
Consider now the association between age and attitude towards
voting for a female President, by constructing a table between
\emph{AGEGROUP} and \emph{FEPRES}. In the population, do people in different
age groups differ in their willingness to vote for a female President?
Interpret the results of the
$\chi^{2}$ test and illustrate your answer with one or two percentages
from the two-way table.
\end{enumerate}

%\textbf{Further practice, NOT part of the homework}: You can of course
%also examine associations between any other variables that you find
%interesting. When doing so, please note that the data set contains at
%least three types of user-missing responses; see \textbf{Values} in the
%Variable View window, and page \pageref{p_missingcodes} for a
%discussion. These are initially included as separate categories in the
%tables, which is fine for exploratory analyses. For final $\chi^{2}$
%tests, however, it is typically better to exclude them by defining their
%codes explicitly as missing values from the \textbf{Missing} column in
%Variable View (for an example, see how the missing value codes have
%already been defined for \emph{FEFAM}).

\newpage


\section[Week 5: Comparing two population means]{WEEK 5 class: Inference for two population means}
\textbf{Data set}:
The data file used today is \textbf{ESS5\_GBFR.sav}. It contains
data for a selection of variables from the 2010 European Social Survey
for respondents in Great Britain and France\footnote{ESS Round 5: European Social
Survey Round 5 Data (2010). Data file edition 2.0. Norwegian Social Science Data Services,
Norway � Data Archive and distributor of ESS data. The full data can be
obtained from \texttt{http://ess.nsd.uib.no/ess/round5/}.}. Only a few of the variables are
used in the exercises; the rest are included in the data set as examples
of the kinds of information obtained from this survey.
%Data files used
%for the homework are \textbf{DRAFT70.SAV} and \textbf{DRAFT71.SAV}.
%explained further in the instructions for the homework.


%These will be
%explained below.

\textbf{Two-sample inference for means in SPSS}

\begin{itemize}
\item
$t$-tests and confidence intervals for two independent samples for
inference on the difference of the population
means: \textbf{Analyze/Compare Means/Independent-Samples T Test}.
The variable of interest $Y$ is placed under \textbf{Test Variable(s)} and
the explanatory variable $X$ under \textbf{Grouping Variable}.
The values of $X$ identifying the two groups being compared
are defined under \textbf{Define Groups}.
\item
\emph{Box plots} for descriptive purposes are obtained from
\textbf{Analyze/Descriptive
Statistics/Explore}.
Here we want to draw side-by-side box plots for values of a
response variable $Y$, one plot for each distinct value of an
explanatory variable $X$. The name of $Y$ is placed under
\textbf{Dependent List} and that of $X$ under \textbf{Factor List}.
Box plots are obtained by
selecting \textbf{Plots/Boxplots/Factor levels
together}.
\item
Tests and confidence intervals for single means (c.f.\ Section \ref{s_means_1sample}) are not
considered today. These are obtained from
\textbf{Analyze/Compare Means/One-Sample T Test}. They
can also be used to carry out inference for
comparisons of means  between two \emph{dependent}  samples
(c.f.\ Section \ref{s_means_dependent}).
\end{itemize}

%\newpage
\textbf{Classwork}\\
Consider the survey data in the file \emph{ESS5\_GBFR.sav}. We will
examine two variables, and carry out statistical inference to
compare their means among
the survey populations of adults in
Great Britain and France\footnote{Strictly speaking, the analysis should
incorporate sampling weights (variable \emph{DWEIGHT}) to adjust for
different sampling probabilities for different types of respondents.
Here the weights are ignored. Using them would not change the main
conclusions for these variables.}.

\begin{enumerate}
\item
The variable \emph{WKHTOT} shows the
number of hours per week the respondent normally works in his
or her main job.
Obtain box plots and descriptive statistics for
this variable separately for each country
(identified by the variable \emph{CNTRY}).
Compare measures of central tendency and variation for \emph{WKHTOT}
between the two countries.
What do you observe?
\item
Obtain a $t$-test and confidence interval for the difference of
weekly working hours between Britain and France
(specify the values of the country variable as
\textbf{Define Groups/Group 1:} \emph{GB} and
\textbf{Group 2:} \emph{FR} as coded in the data).
Details of SPSS output for this are explained in Chapter \ref{c_means};
you can use the results under
the assumption of equal population variances.
What do you conclude?
Is there a statistically significant difference in the average values
of \emph{WKHTOT} between the two countries? What does the confidence
interval suggest about the size of the difference?
\item
The variable \emph{STFJBOT} asks those in paid work,
``How satisfied are you with the balance between the
time you spend on your paid work and the time you spend on
other aspects of your life?''. Respondents are asked to rate their
level of satisfaction on a scale from 0-10, where
0 means ``Extremely dissatisfied'' and 10 means ``Extremely satisfied''.
Repeat exercises 1 and 2 for this variable, and
compare also histograms of \emph{STFJBOT} for each country. What do you
observe?
\end{enumerate}


\textbf{HOMEWORK}

\begin{enumerate}
\item
Write up your answers to the second class exercise, answering these specific
questions:
\begin{enumerate}
\item
What are the observed sample means for \emph{WKHTOT} for French and British respondents?
\item
Is there a statistically significant difference in the average values
of \emph{WKHTOT} between the two countries? State the value of the test statistic and its
corresponding $P$-value. You may assume equal population variances for this test.
\item
Interpret the 95\% confidence interval for the difference.
\end{enumerate}
\item
The variable \emph{WKHSCH} asks respondents,
``How many hours a week, if any, would you choose to work, bearing in
mind that your earnings would go up or down according to
how many hours you work?''. Is there a statistically significant difference between ideal
(rather than actual) work hours for French and British respondents? Carry out a t-test
and report and interpret the results.
\item
The variable \emph{STFMJOB} asks respondents,
``How satisfied are you in your main job?''. Respondents are asked to rate their
level of satisfaction on a scale from 0-10, where
0 means ``Extremely dissatisfied'' and 10 means ``Extremely satisfied''. Is there a statistically
significant difference, at the 5\% level of significance, between mean levels of job satisfaction
for French and British respondents? Answer this question by using the 95\% confidence interval for
the difference in means (you need the full t-test output to obtain the confidence interval, but you
need not report the results of the t-test itself for this question).
\end{enumerate}



\newpage
$\;$
\newpage


\section[Week 7: Inference for proportions]{WEEK 7 class: Inference for population proportions}

\textbf{Data sets}: Files
\textbf{BES2010post\_lastdebate.sav} and
\textbf{BES2010pre\_lastdebate.sav}.


\textbf{Inference on proportions in SPSS}
%The following new procedures are needed in today's class:
\begin{itemize}
\item
SPSS menus do not provide procedures for calculating the tests
and confidence intervals for proportions discussed in Chapter \ref{c_probs}.
This is not a serious limitation, as the calculations are quite simple.
\item
It is probably easiest to use a pocket calculator for the calculations, and
this is the approach we recommend for this class.
The only part of the analysis it cannot do is calculating
the precise $P$-value for the tests, but even this can be avoided by
using critical values from a statistical table such as the one at the
end of this Coursepack to determine approximate $P$-values (or by using
an online $P$-value calculator --- see ``About Week 4 class'' on the
Moodle page for suggested links).
\end{itemize}

\subsubsection{Classwork}

The survey data set \emph{BES2010post\_lastdebate.sav} contains part of the
information collected by the British Election Study, an ongoing
research programme designed to understand voter choices in the UK
\footnote{The data can be obtained from
\texttt{http://bes2009-10.org/}, which gives further information on
the survey, including the full text of the questionnaires. The data
analysed in this class and homework are from the BES Campaign Internet Panel Survey,
which has been divided into two data sets corresponding to two time periods
leading up to the General Election.}.

In the run-up to the UK General Election on 6 May 2010, opinion polls reported quite dramatic
changes in popularity of the Liberal Democrat party. Key to their increasing popularity was the
 performance of their party leader, Nick Clegg, in a series of three televised debates
between the leaders of the three main political parties (the other participants were Gordon Brown
for Labour and David Cameron for the Conservative party). The debates were broadcast between
15 and 29 April 2010.

The data in \emph{BES2010post\_lastdebate.sav} contain information on respondents' voting intentions,
obtained after the debates had ended (i.e.\ between 30 April and 6 May).

%last week, and consider two dichotomous variables.
%For the first
%of them, one-sample inference for the population proportion in Britain
%will be considered (exercise 1). For the second one,
%proportions in Britain and France will be compared (exercise 2).

\begin{enumerate}
\item
\emph{VOTE\_LIBDEM} is a dichotomous variable indicating whether a
respondent intended to vote for the Liberal Democrats (value 1) or some other
party (0) in the 2010 General Election. The value of this variable is by
definition missing for those who had not decided which way they would vote or who did
not intend to vote at all, so they
are automatically excluded from the analysis. The parameter of interest $\pi$ is now the
population proportion of those who \emph{say} they would vote Liberal Democrat. We
will compare it to 0.23, the proportion of the vote the party actually
received in 2010. The analysis is thus one-sample inference on a
population proportion, and the relevant formulas are (\ref{ztestp}) on p.\ \pageref{ztestp} for
the test statistic and (\ref{cip2}) on p.\ \pageref{cip2} for the
confidence interval.
\begin{itemize}
\item
Begin by creating a frequency table of
\emph{VOTE\_LIBDEM}. This should show that the sample estimate of $\pi$ is
0.260, out of $3226$ non-missing responses.
Thus $n=3226$ and $\hat{\pi}=0.260$ in the notation of Chapter
\ref{c_probs}.
\item
For the one-sample significance test, the value of $\pi$ under the
null hypothesis is $\pi_{0}=0.230$. Using equation (\ref{ztestp}), the
value of the test statistic $z$ is thus given by the calculation
\[
z = \frac{0.260-0.230}{\sqrt{0.230\times (1-0.230)/3226}}
\]
Calculate this using a calculator.
The result should be $z=4.049$.
\item
The (two-sided) $P$-value for this is the probability that a value from
the standard normal distribution is at most $-4.049$ or at least 4.049.
Evaluate this approximately by comparing the value of $z$ to
critical values from the standard normal distribution (c.f.\ Table
\ref{t_ttable} on p.\ \pageref{t_ttable}) as explained in Section
\ref{ss_probs_test1sample_samplingd}.
Here, for example, $z$ is larger than 1.96, so the two-sided $P$-value
must be smaller than 0.05. Convince yourself that you understand this
statement.
\item
Calculate a 95\% confidence interval for the population proportion of
prospective Liberal Democrat voters, using equation
(\ref{cip2}).
\end{itemize}
What do you conclude about the
proportions of prospective and actual Liberal Democrat voters? Why might the two
differ from each other?
%\item
%The variable \emph{VOTEFN} indicates whether a French
%respondent reported voting for the
%far-right National Front (FN) party (1) or some other party (0) in
%the first round of the 2002 parliamentary election.
%Repeat exercise 1 for this variable, now
%comparing the proportion of self-reported FN voters to the party's actual
%share of the vote, which was 11.2\%
\item
The variable
\emph{TVDEBATE} indicates  whether the respondent reports having
 watched any of the three televised debates (1 for Yes, at least one watched,
0 otherwise - this includes ``no'' and ``don't know'' responses). We will compare the
proportion of people intending to vote Liberal Democrat amongst those who
watched some or all of the debates with those who did not, using the two-sample methods of
analysis discussed in Section \ref{s_probs_2samples}. The formula of the
$z$-test statistic for testing the hypothesis of equal population
proportions is thus (\ref{ztestDpi}) on page \pageref{ztestDpi}, and a
confidence interval for the difference of the porportions is
(\ref{ciDpi}).
\begin{itemize}
\item
Begin by calculating the relevant
sample proportions. The easiest way to do this is by creating a
two-way contingency table between \emph{TVDEBATE} and
\emph{VOTE\_LIBDEM} as you did in the Week 2 and 4 classes. The results required for the analysis
considered here are all shown in the resulting table. Convince yourself
that these show that, in the notation of Section \ref{s_probs_2samples},
\begin{itemize}
\item
$n_{1}=930$ and $\hat{\pi}_{1}=0.218\; (=203/930)$,
\item
$n_{2}=2296$ and $\hat{\pi}_{2}=0.277\; (=636/2296)$,
\end{itemize}
where 1 denotes respondents who did not watch any of the
debates and 2 those who watched at least some.
The pooled estimated proportion $\hat{\pi}$ (formula
\ref{phat2sample}) used in the test statistic (\ref{ztestDpi}) is here
$\hat{\pi}=0.260$, shown on the ``Total'' row.
\item Calculate the test
statistic, its $P$-value and a 95\%  confidence for the difference in
population proportions, using the relevant formulas. For example, the
test statistic is here given by
\[
z= \frac{0.277-0.218}{\sqrt{0.260\times (1-0.260)\times
(1/2296+1/930)}}.
\]
\end{itemize}
What do you conclude? Is there evidence that those who watched at least
some of the leaders' debates were more likely to declare an intention to
vote Liberal Democrat? If there is,
how big is the difference in proportions of prospective Liberal Democrat
voters between the debate-watchers and debate-non-watchers? \end{enumerate}

%\newpage
\subsubsection{HOMEWORK}

Write up your answers to the second class exercise. In particular, answer the following
specific questions:
\begin{enumerate}
\item
What proportion of respondents say that they did watch at least some of the
leaders' debates? And what proportion did not? Of those who watched at
least some of the leaders' debates, what proportion said they intended
to vote Liberal Democrat? And what proportion of those who did
\emph{not} watch any of the leaders' debates said they intended to vote
Liberal Democrat?
\item
Calculate the test statistic and find its corresponding approximate $P$-value for the difference in
population proportions of prospective Liberal Democrat voters among those who did and did not
watch the leaders' debates. Show your working.
State the conclusion from the test.
\item
Calculate a 95\% confidence interval around this difference. State its lower and upper limits.
\item
Write a brief substantive interpretation of your results.
\end{enumerate}

The data set \emph{BES2010pre\_lastdebate.sav} contains responses to the
same question - whether respondents intended to vote Liberal Democrat or not - but
asked before the last of the party leaders' debates.
Repeat the analysis you carried out for the first class exercise, but using this
data set. In other words carry out a one-sample analysis,
of the kind done in exercise 1 above, to compare
the proportion of respondents who said they intended to vote Liberal Democrat with
the proportion who actually did. Answer the following questions:
\begin{enumerate}
\item
State the null hypothesis for the test.
\item
Calculate the test statistic and find its corresponding approximate $P$-value. Show your workings.
\item
Give a brief interpretation of the results. Do they differ from the other data set? Can you think
of any reasons for this? (This last question invites some speculation - do not worry if you don't
have any ideas! But see the sample answer if you are interested in our speculation.)
\end{enumerate}

\newpage

\section[Week 8: Correlation and simple linear regression 1]{WEEK 7 class: Correlation and simple linear regression 1}

\textbf{Data set}: Files \textbf{decathlon2012.sav}.

\subsubsection{Scatterplots, correlation and simple linear
regression in SPSS}

\begin{itemize}
\item
A scatterplot is obtained from \textbf{Graphs/Legacy
Dialogs/``Scatter/Dot''/ Simple Scatter/Define}.
The variables for the $X$-axis and $Y$-axis are placed in the \textbf{X
Axis} and \textbf{Y Axis} boxes respectively. Double-clicking on the
plot in the Output window opens it in a \emph{Chart Editor}, where
various additions to the graph can be made. A fitted
straight line is added from \textbf{Elements/Fit Line at Total}.
A least squares fitted line is the default under this option, so it is
drawn immediately and you can just click \textbf{Close}. Closing the
Chart Editor commits the changes to the Output window.
\item
A correlation matrix is obtained from
\textbf{Analyze/Correlate/Bivariate}, when \textbf{Correlation
Coefficients/Pearson} is selected (which is the default, so you should
not need to change it). The variables included in the correlation
matrix are placed into the \textbf{Variables} box. The output also
includes a test for the hypothesis that the population correlation
is 0, but we will ignore it.
\item
Linear regression models are obtained from
\textbf{Analyze/Regression/Linear}. The response variable is placed
under \textbf{Dependent} and the explanatory variable under
\textbf{Independent(s)}. The dialog box has many options for various
additional choices. Today you can leave all of them at their default
values, except that you should select \textbf{Statistics/Regression
Coefficients/Confidence intervals} to include also 95\% confidence intervals
for the regression coefficients in the output.
\end{itemize}

\subsubsection{Classwork}

Decathlon is a sport where the participants complete ten different
athletics events over two days. Their results in each are then
translated into points, and the winner is the competitor with the
highest points total for the ten events. The file \emph{decathlon2012.sav} contains
the results of the decathlon competition at the 2012 Olympics in London
for the 26 athletes who finished the competition.\footnote{Official
results obtained from
\texttt{www.olympic.org/london-2012-summer-olympics}.}
The results for each
event are given both in their original units (variables with names
beginning with ``mark\_'') and in decathlon points (names beginning with
``points\_''). The ten events are identified by the variable labels in Variable View.
The variable \emph{points\_total} gives the final points total for each competitor.

\newpage
\begin{enumerate}
\item
Create a scatterplot between the result ($X$-axis) and points ($Y$-axis)
for one event, the 100-metre sprint (variables \emph{MARK\_100M} and
\emph{POINTS\_100M}), and add a fitted line. This simply provides information on
the calculation used to transform the result into points. Clearly a
linear calculation is used for this, at least over the range
of results in these data. Notice the downward slope of the line: the faster the
result, the higher the number of points. From now on, for simplicity we will
consider only the points variables for each event.
\item
Obtain the correlation matrix for all pairs of variables among the ten
individual points scores and the total score. Consider first correlations
between the individual events only. Which correlations
tend to be high (say over 0.5), which ones close to zero and which ones
even negative? Can you think of any reasons for this? Draw scatterplots
and fitted lines for a few pairs of variables with different sizes of
correlations (here the variables are treated symmetrically, so it does
not matter which one is placed on the $X$-axis). Can these associations
be reasonably described as linear?
\item
Consider now the correlations between the ten event scores and the final
score \emph{POINTS\_TOTAL}. Which of them is highest, and which one lowest?
Examine the scatterplot and fitted line between points for 100 metres
(\emph{POINTS\_100M}) and the total score (POINTS\_TOTAL). Fit a line
to this scatterplot variables, with \emph{POINTS\_100M} as the
explanatory variable. Interpret the results. Does there appear to be an
 association between the points for 100 metres and the total score? What is the
 nature of the association?

Suppose you were told that a competitor received 800 points (a time of
about 11.3 seconds) for 100 metres, the first event of the decathlon.
Based on the fitted model, what final points score would you predict for
him? You can calculate this fitted value with a pocket calculator.
What would be the predicted value if the 100-metre score was
950 points (about 10.6 s) instead?
\end{enumerate}

\textbf{HOMEWORK}

\begin{enumerate}

\item
Briefly discuss the correlation matrix produced in the class. Pick out a few
examples for illustration - which correlations are highest, and which ones lowest,
and which ones negative? You may comment on correlations between individual events, as
well as on correlations between the final score and individual events.
\item
Obtain the scatterplot and linear regression model for
the total score given points for the long jump, one of the
field events (POINTS\_LONGJUMP). Is the
score for long jump strongly or weakly associated with the final score?
Interpret the slope coefficient. Suppose you were told that a competitor received
900 points (a jump of about 7.4 metres) for the long jump.
Based on the fitted model, what final points score would you predict for
him?
\item
Obtain the scatterplot and linear regression model for
the total score given points for throwing the discus, another of the
field events (POINTS\_DISCUS). Interpret the slope coefficient. Is the
score for discus strongly or weakly associated with the final score?
\end{enumerate}


\newpage

\section[Week 9: Simple linear regression 2]{WEEK 8 class: Simple linear
regression and 3-way tables}

\textbf{Data set}: File \textbf{GSS2010.SAV}. This contains a
selection of variables on attitudes and demographic characteristics for
2044 respondents in the 2010 U.S.\ General Social Survey
(GSS)\footnote{The data can be obtained from
\texttt{www3.norc.org/GSS+Website/}, which gives further information on
the survey, including the full text of the questionnaires. }. Only a few
of the variables are used in the exercises.

\subsubsection{Classwork - linear regression}

Here we will focus on the variables
\emph{EDUC}, \emph{PAEDUC}, \emph{MAEDUC} and \emph{SPEDUC}. These
show the number of years of education completed by, respectively, the
survey respondent him/herself, and the respondent's father, mother and
spouse.
\begin{enumerate}

\item
Obtain basic descriptive statistics for the variables.
Here they can be compared directly, because the meaning of the variable
is similar in each case. We can even draw side-by-side box plots
for the variables (rather than for values of a single variable at
different levels of another, as before). These can be obtained from
\textbf{Analyze/Descriptive Statistics/Explore} by placing all the
variables under \textbf{Dependent List} and selecting
\textbf{Plots/Boxplots/Dependents together}. You should then also select
\textbf{Options/Missing Values/Exclude cases pairwise} to include all
non-missing values for each variable (here \emph{SPEDUC} has for obvious
reasons more missing values than the others).
\item
Obtain the correlation matrix of the four variables. Which
correlations are highest, and which ones lowest?
\item
Draw a scatterplot with fitted line
for \emph{EDUC} given \emph{PAEDUC}.
%the respondent's education given
%his or her father's education.
Fit a linear regression
model between these variables, regressiong \emph{EDUC} (response variable)
on \emph{PAEDUC} (explanatory variable).
Interpret the results. Is there a statistically significant linear association between
a person's years of schooling and those of his/her father? Interpret the estimated
regression coefficient, $t$-statistic and
$P$-value, and 95 per cent confidence interval.
\item
Based on
the fitted model, what is the predicted number of years of education for
a respondent whose father completed 12 years of education?
\end{enumerate}

\subsubsection{HOMEWORK: Simple linear regression and three-way tables}

The homework exercise uses the same data set for two different types of
analysis.

%\newpage
\subsubsection{Linear regression}
Draw a scatterplot with fitted line
for \emph{EDUC} given \emph{MAEDUC}.
Fit a linear regression
model between these variables, regressiong \emph{EDUC} (response variable)
on \emph{MAEDUC} (explanatory variable).
\begin{enumerate}

\item
Interpret the results: Is there a statistically significant linear association between
a person's years of schooling and those of his/her mother? Interpret the estimated
regression coefficient, $t$-statistic and
$P$-value, and 95 per cent confidence interval.
\item
Based on the fitted model, what is the predicted number of years of education for
a respondent whose mother completed 10 years of education?
\item
Interpret the R-squared statistic for the model.
\end{enumerate}

\subsubsection{Analysing multiway contingency tables in SPSS}

Three-way contingency tables are again obtained from
\textbf{Analyze/Descriptive Statistics/Crosstabs}. The only change
from Week 4 class is that the conditioning variable is now placed in
the \textbf{Layer 1 of 1} box. This produces a series of partial two-way
tables between the row and column variables specified in the
\textbf{Row(s)} and \textbf{Column(s)} boxes, one for each category of
the \textbf{Layer} variable. Percentages and $\chi^{2}$ test are similarly
calculated separately for each partial table.
%Tables with more than three variables are obtained
%by placing additional variables into further \textbf{Layer} boxes,
%accessed by clicking on \textbf{Next} in the \textbf{Layer 1 of 1} box.
%This simply produces more partial tables, one for each combination of
%the categories of the conditioning (layer) variables. We will, however,
%not consider this possibility today.

For this example we elaborate on the first two exercises from Week 4 class. To remind
you, the categorical variables we are analysing are these:
\begin{itemize}
\item
The respondent's sex, recorded as the variable \emph{SEX}.
\item
age as \emph{AGEGROUP} in three groups: 18-34, 35-54 and 55 or over.
\item
\emph{FEFAM}: Level of agreement with the following statement: ``It is much better for
everyone involved if the man is the achiever outside the home and the woman takes care
of the home and family'', with response options Strongly agree, Agree,
Disagree, and Strongly disagree.
\end{itemize}

\begin{enumerate}
\item
First remind yourself of the associations between SEX and FEFAM and between
AGEGROUP and FEFAM. Obtain the two-way contingency
table between \emph{FEFAM} and \emph{SEX}, including appropriate
percentages and $\chi^{2}$ test of independence. Repeat the procedure
for \emph{FEFAM} by \emph{AGEGROUP}. What do you learn about
the associations between attitude and sex, and between attitude and age?
\item
Sociologists would suggest that the relationship between sex and attitude towards
male and female work roles might be different for different age groups. In other
words, age might modify the association between sex and attitude. Investigate this
possible interaction between the three variables.
Create a three-way table where \emph{FEFAM} is the column
variable, \emph{SEX} the row variable  and \emph{AGEGROUP} the layer
(conditioning) variable. Study the SPSS output, and make sure you
understand how this shows three partial tables of \emph{FEFAM} vs.\
\emph{SEX}, one for each possible value of \emph{AGEGROUP}. Examine and
interpret the associations in the three partial tables. State the results of the
$\chi^{2}$ test for each partial table, and illustrate your interpretations with some
appropriate percentages. Finally, summarise your findings: are there
differences in the nature, strength or significance of the association
between sex and attitude, depending on the age group? Comment on how this interpretation
differs from the initial two-way table of \emph{FEFAM} and \emph{SEX}.
\end{enumerate}

\newpage

\section[Week 10: Multiple linear regression]{WEEK 9 class: Multiple linear regression}

\textbf{Data set}: File \textbf{humandevelopment2011.sav}.


\subsubsection{Multiple linear regression in SPSS}

\begin{itemize}
\item
Multiple linear regression is obtained from
\textbf{Analyze/Regression/Linear}, by placing all of the
required explanatory variables in the \textbf{Independent(s)} box. No
other changes from last week are required.
\item
To include categorical explanatory variables, the necessary
dummy variables have to be created first. The ones for today's
class are already included in the data set. If you need to create dummy
variables for your own analyses in the future, it is usually
easiest to do so from \textbf{Transform/Compute Variable}.
Some of the buttons on the keypad shown
in that dialog box are \emph{logical operators} for defining conditions for which the outcome is either 1
(True) or 0 (False), as required by a dummy variable. For example, the
categorical variable \emph{INCOME\_GROUP} in today's data set has the value 3
if the country is in the high income group. The dummy variable \emph{HIGH\_INCOME} was created
from this by entering \textbf{Target Variable:} \emph{HIGH\_INCOME} and \textbf{Numeric
Expression:} \emph{INCOME\_GROUP=3}. This means that the new
variable \emph{HIGH\_INCOME} will have the value 1 for countries for which
\emph{INCOME\_GROUP} is equal to 3, and will be 0 otherwise.
Other logical operators may also be used: for example,
\emph{urban\_pop$<$50} would produce 1 if the variable
\emph{URBAN\_POP} was less than
50 and 0 otherwise.
\end{itemize}


\subsubsection{Classwork}


The file \emph{humandevelopment2011.sav} contains data on a number of indicators of
what might broadly be called development, for 194 countries in 2011. These were collated
from two international data agency sources\footnote{United Nations Development
Programme \emph{International Human Development Indicators},
\texttt{http://hdr.undp.org/en/data/};
World Bank \emph{Worldwide Governance Indicators},
\texttt{http://info.worldbank.org/governance/wgi/pdf/wgidataset.xlsx};
World Bank \emph{World Development Indicators},
\texttt{http://data.worldbank.org/indicator/SP.DYN.IMRT.IN}.
}.
The response
variable considered today is \emph{SCHOOL\_YEARS}, which records for each
country the mean number of years of schooling taken by the adult population.
We treat it here as a general indicator
of the educational situation in a country, which is an important aspect of development.
We will consider the following
explanatory variables for it:
\begin{itemize}

\item
\emph{URBAN\_POP}: the degree of urbanisation of the country, specifically the
percentage of the country's population living in urban areas
variable
\item
\emph{GOVERNANCE}, a continuous variable contructed from expert opinion surveys to reflect the
perceived effectiveness of government in delivering services.
%\footnote{The definition of this indicator,
%from the World Bank's Governance Indicators, is as follows: ``Reflects perceptions
%of the quality of public services, the quality of the civil service and the degree
%of its independence from political pressures, the quality of policy formulation and
%implementation, and the credibility of the government's commitment to such policies.''}
\item
\emph{INFANT\_MORTALITY}, number of infants dying before 1 year old, per
1,000 live births --- a
``proxy'' indicator representing the health of the population
\item
\emph{INCOME\_GROUP}, classified as low, middle or high income
economies.
% \footnote{grouping economies
%according to 2011 Gross National Income per capita, calculated using the World Bank Atlas method.
%The groups are: low income, \$1,025 or less; middle income, \$1,026 - \$12,475; and high income,
%\$12,476 or more.}.
This is also provided in the form of three dummy variables: \emph{LOW\_INCOME},
 \emph{MIDDLE\_INCOME} and \emph{HIGH\_INCOME}.
\end{itemize}

\begin{enumerate}
\item
Obtain some descriptive statistics for the continuous variables, to gain
and impression of their ranges. A quick way of doing this is via
\textbf{Analyze/Descriptive Statistics/Frequencies}, unchecking the ``Display frequency tables''
and requesting minimum and maximum values.
\item
Investigate the idea that increased urbanisation is linked to greater availability of
schooling for people. Obtain a scatterplot and a simple linear
regression model for
\emph{SCHOOL\_YEARS} given \emph{URBAN\_POP}.
What do you observe in the scatterplot? Interpret the regression output.
\item
Now consider the possibility that schooling may also be explained by the effectiveness
of governments in providing public services (such as education). Fit a
multiple linear regression model for \emph{SCHOOL\_YEARS} given both
\emph{URBAN\_POP} and \emph{GOVERNANCE}.
Compare the the estimated coefficient of
\emph{URBAN\_POP} for this model with the coefficient of the same
variable in the model in Question 2.
What do you conclude? Does the
association between schooling and urbanisation change when we control for
government effectiveness? If so, in what way? Interpret the estimate coefficient of \emph{GOVERNANCE}
in the fitted model, the results of its $t$-test and its 95\% confidence interval.
\item
Next consider the possible explanatory value of the income wealth of a country
for understanding variation in schooling years. Include income by entering two
of the three dummy variables for income group. For the most convenient interpretation,
we suggest that you leave ``low income'' as the reference group, and enter the
dummies for \emph{MIDDLE\_INCOME} and \emph{HIGH\_INCOME} in the model.
Interpret the values of the estimated regression coefficients for the two income dummy variables. In
addition, for each one state the null hypothesis for
its $t$-test, and interpret the result of the test and 95\% confidence intervals.
\item
Using this model, what level of schooling would you predict for a country with 70\% urban population,
a score of 1.5 on governance, and a high income economy?
\item
Using this model, what level of schooling would you predict for a country with 30\% urban population,
a score of -0.2 on governance, and a low income economy?
\end{enumerate}

\textbf{HOMEWORK}
\begin{enumerate}
\item
Write up your answers to the last three questions in the class exercise.
\item
Finally, consider one more possible explanatory variable:
\emph{INFANT\_MORTALITY}. Add this variable to the
multiple linear regression model fitted above. Is it statistically significant, at the 1\% level of significance?
Interpret the value of its estimated coefficient, and its 95\% confidence interval. Take care to make sense of
the sign (positive or negative) of the coefficient.
\item
Has the inclusion of \emph{INFANT\_MORTALITY} modified the interpretation of any of the other explanatory variables
in the model? Are they all statistically significant, at the 5\% level of significance?
Briefly outline the similarities and differences between the results for this final model and
the model fitted in the class exercise.
\end{enumerate}


\newpage
\section[Week 11: Review and Multiple linear regression]{WEEK 10 class:
Review and Multiple linear regression}

\textbf{Data set}: File \textbf{ESS5GB\_trust.sav}.

This class is for you to revisit any topics of your choosing. Make the most
of the opportunity to ask your class teachers any questions you have about any of the
course material, and to practise any of the analyses you have learned during the course.

As an optional exercise, the data file \textbf{ESS5GB\_trust.sav} is provided. This contains a selection
of variables from the survey of British respondents that forms the 2010 wave of the European Social
Survey\footnote{ESS Round 5: European Social Survey Round 5 Data (2010). Data file edition 2.0. Norwegian Social Science Data Services, Norway � Data Archive and distributor of ESS data. The full data can be
obtained from \texttt{http://ess.nsd.uib.no/ess/round5/}.}.

We suggest that you use the data to practise multiple linear regression modelling on one or more of the variables
capturing people's levels of trust in institutions. For these questions, respondents were asked the
following: ``Using this card, please tell me on a score of 0-10 how much you personally trust each
of the institutions I read out. 0 means you do not trust an institution at all, and 10 means you
have complete trust.''
The institutions (and their variable names) are:
\begin{itemize}
\item
\emph{trstprl}: Trust in country's parliament
\item
\emph{trstlgl}: Trust in the legal system
\item
\emph{trstplc}: Trust in the police
\item
\emph{trstplt}: Trust in politicians
\item
\emph{trstprt}: Trust in political parties
\item
\emph{trstep}: Trust in the European Parliament
\item
\emph{trstun}: Trust in the United Nations
\end{itemize}

After you choose a response variable that interests you, you will need to select some potential
explanatory variables to test. The data set contains a number of variables. Some are socio-demographic,
such as age and gender.
Some are attitudinal or behavioural, such as amount of time spent reading newspapers. You will need to make a
judgement about the levels of measurement of the variables, and how to enter them into the model. Use the ``Values''
column in the SPSS Variable View to check how each variable is coded. Note: we suggest
that it is not too much of a compromise to treat the variables on television, radio and newspaper consumption
as continuous, interval level variables. Note also: we have provided dummy variables for the
categorical variables in the data set.






\textbf{HOMEWORK}

As this is the last week of the course, there is no homework. You can
find further information on this and the other class exercises and
homeworks in the model answers, which will be posted in the Moodle site.

%\newpage
%$\; $
\newpage
%\clearpage
%\vspace*{-10cm}
\chapter{
Statistical tables}
\label{c_disttables}



Explanation of the ``Table of standard normal tail probabilities'' on
page \pageref{s_disttables_Z}:
\begin{itemize}
\item
The table shows, for values of $Z$ between 0 and 3.5, the probability
that a value from the standard normal distribution is \emph{larger than}
$Z$ (i.e.\ the ``right-hand'' tail probabilities).
\begin{itemize}
\item
For example, the probability of values larger than 0.50 is 0.3085.
\end{itemize}
\item
For negative values of $Z$, the probability of values \emph{smaller
than} $Z$ (the ``left-hand'' tail probability) is equal to the right-hand
tail probability for the corresponding positive value of $Z$.
\begin{itemize}
\item
For example, the probability of values smaller than $-0.50$ is also 0.3085.
\end{itemize}
\end{itemize}


\newpage
\section{Table of standard normal tail probabilities}
\label{s_disttables_Z}

{\small
\begin{tabular}{|ll|ll|ll|ll|ll|ll|}\hline
$z$ & Prob.\ & $z$ & Prob.\ & $z$ & Prob.\ & $z$ & Prob.\ & $z$ & Prob.\ & $z$ & Prob.\ \\ \hline
 0.00 & 0.5000 & 0.50 & 0.3085& 1.00 & 0.1587& 1.50 & 0.0668& 2.00 & 0.0228&2.50 &0.0062\\
 0.01 & 0.4960 & 0.51 & 0.3050& 1.01 & 0.1562& 1.51 & 0.0655& 2.01 & 0.0222&2.52 &0.0059\\
 0.02 & 0.4920 & 0.52 & 0.3015& 1.02 & 0.1539& 1.52 & 0.0643& 2.02 & 0.0217&2.54 &0.0055\\
 0.03 & 0.4880 & 0.53 & 0.2981& 1.03 & 0.1515& 1.53 & 0.0630& 2.03 & 0.0212&2.56 &0.0052\\
 0.04 & 0.4840 & 0.54 & 0.2946& 1.04 & 0.1492& 1.54 & 0.0618& 2.04 & 0.0207&2.58 &0.0049\\
 0.05 & 0.4801 & 0.55 & 0.2912& 1.05 & 0.1469& 1.55 & 0.0606& 2.05 & 0.0202&2.60 &0.0047\\
 0.06 & 0.4761 & 0.56 & 0.2877& 1.06 & 0.1446& 1.56 & 0.0594& 2.06 & 0.0197&2.62 &0.0044\\
 0.07 & 0.4721 & 0.57 & 0.2843& 1.07 & 0.1423& 1.57 & 0.0582& 2.07 & 0.0192&2.64 &0.0041\\
 0.08 & 0.4681 & 0.58 & 0.2810& 1.08 & 0.1401& 1.58 & 0.0571& 2.08 & 0.0188&2.66 &0.0039\\
 0.09 & 0.4641 & 0.59 & 0.2776& 1.09 & 0.1379& 1.59 & 0.0559& 2.09 & 0.0183&2.68 &0.0037\\
 0.10 & 0.4602 & 0.60 & 0.2743& 1.10 & 0.1357& 1.60 & 0.0548& 2.10 & 0.0179&2.70 &0.0035\\
 0.11 & 0.4562 & 0.61 & 0.2709& 1.11 & 0.1335& 1.61 & 0.0537& 2.11 & 0.0174&2.72 &0.0033\\
 0.12 & 0.4522 & 0.62 & 0.2676& 1.12 & 0.1314& 1.62 & 0.0526& 2.12 & 0.0170&2.74 &0.0031\\
 0.13 & 0.4483 & 0.63 & 0.2643& 1.13 & 0.1292& 1.63 & 0.0516& 2.13 & 0.0166&2.76 &0.0029\\
 0.14 & 0.4443 & 0.64 & 0.2611& 1.14 & 0.1271& 1.64 & 0.0505& 2.14 & 0.0162&2.78 &0.0027\\
 0.15 & 0.4404 & 0.65 & 0.2578& 1.15 & 0.1251& 1.65 & 0.0495& 2.15 & 0.0158&2.80 &0.0026\\
 0.16 & 0.4364 & 0.66 & 0.2546& 1.16 & 0.1230& 1.66 & 0.0485& 2.16 & 0.0154&2.82 &0.0024\\
 0.17 & 0.4325 & 0.67 & 0.2514& 1.17 & 0.1210& 1.67 & 0.0475& 2.17 & 0.0150&2.84 &0.0023\\
 0.18 & 0.4286 & 0.68 & 0.2483& 1.18 & 0.1190& 1.68 & 0.0465& 2.18 & 0.0146&2.86 &0.0021\\
 0.19 & 0.4247 & 0.69 & 0.2451& 1.19 & 0.1170& 1.69 & 0.0455& 2.19 & 0.0143&2.88 &0.0020\\
 0.20 & 0.4207 & 0.70 & 0.2420& 1.20 & 0.1151& 1.70 & 0.0446& 2.20 & 0.0139&2.90 &0.0019\\
 0.21 & 0.4168 & 0.71 & 0.2389& 1.21 & 0.1131& 1.71 & 0.0436& 2.21 & 0.0136&2.92 &0.0018\\
 0.22 & 0.4129 & 0.72 & 0.2358& 1.22 & 0.1112& 1.72 & 0.0427& 2.22 & 0.0132&2.94 &0.0016\\
 0.23 & 0.4090 & 0.73 & 0.2327& 1.23 & 0.1093& 1.73 & 0.0418& 2.23 & 0.0129&2.96 &0.0015\\
 0.24 & 0.4052 & 0.74 & 0.2296& 1.24 & 0.1075& 1.74 & 0.0409& 2.24 & 0.0125&2.98 &0.0014\\
 0.25 & 0.4013 & 0.75 & 0.2266& 1.25 & 0.1056& 1.75 & 0.0401& 2.25 & 0.0122&3.00 &0.0013\\
 0.26 & 0.3974 & 0.76 & 0.2236& 1.26 & 0.1038& 1.76 & 0.0392& 2.26 & 0.0119&3.02 &0.0013\\
 0.27 & 0.3936 & 0.77 & 0.2206& 1.27 & 0.1020& 1.77 & 0.0384& 2.27 & 0.0116&3.04 &0.0012\\
 0.28 & 0.3897 & 0.78 & 0.2177& 1.28 & 0.1003& 1.78 & 0.0375& 2.28 & 0.0113&3.06 &0.0011\\
 0.29 & 0.3859 & 0.79 & 0.2148& 1.29 & 0.0985& 1.79 & 0.0367& 2.29 & 0.0110&3.08 &0.0010\\
 0.30 & 0.3821 & 0.80 & 0.2119& 1.30 & 0.0968& 1.80 & 0.0359& 2.30 & 0.0107&3.10 &0.0010\\
 0.31 & 0.3783 & 0.81 & 0.2090& 1.31 & 0.0951& 1.81 & 0.0351& 2.31 & 0.0104&3.12 &0.0009\\
 0.32 & 0.3745 & 0.82 & 0.2061& 1.32 & 0.0934& 1.82 & 0.0344& 2.32 & 0.0102&3.14 &0.0008\\
 0.33 & 0.3707 & 0.83 & 0.2033& 1.33 & 0.0918& 1.83 & 0.0336& 2.33 & 0.0099&3.16 &0.0008\\
 0.34 & 0.3669 & 0.84 & 0.2005& 1.34 & 0.0901& 1.84 & 0.0329& 2.34 & 0.0096&3.18 &0.0007\\
 0.35 & 0.3632 & 0.85 & 0.1977& 1.35 & 0.0885& 1.85 & 0.0322& 2.35 & 0.0094&3.20 &0.0007\\
 0.36 & 0.3594 & 0.86 & 0.1949& 1.36 & 0.0869& 1.86 & 0.0314& 2.36 & 0.0091&3.22 &0.0006\\
 0.37 & 0.3557 & 0.87 & 0.1922& 1.37 & 0.0853& 1.87 & 0.0307& 2.37 & 0.0089&3.24 &0.0006\\
 0.38 & 0.3520 & 0.88 & 0.1894& 1.38 & 0.0838& 1.88 & 0.0301& 2.38 & 0.0087&3.26 &0.0006\\
 0.39 & 0.3483 & 0.89 & 0.1867& 1.39 & 0.0823& 1.89 & 0.0294& 2.39 & 0.0084&3.28 &0.0005\\
 0.40 & 0.3446 & 0.90 & 0.1841& 1.40 & 0.0808& 1.90 & 0.0287& 2.40 & 0.0082&3.30 &0.0005\\
 0.41 & 0.3409 & 0.91 & 0.1814& 1.41 & 0.0793& 1.91 & 0.0281& 2.41 & 0.0080&3.32 &0.0005\\
 0.42 & 0.3372 & 0.92 & 0.1788& 1.42 & 0.0778& 1.92 & 0.0274& 2.42 & 0.0078&3.34 &0.0004\\
 0.43 & 0.3336 & 0.93 & 0.1762& 1.43 & 0.0764& 1.93 & 0.0268& 2.43 & 0.0075&3.36 &0.0004\\
 0.44 & 0.3300 & 0.94 & 0.1736& 1.44 & 0.0749& 1.94 & 0.0262& 2.44 & 0.0073&3.38 &0.0004\\
 0.45 & 0.3264 & 0.95 & 0.1711& 1.45 & 0.0735& 1.95 & 0.0256& 2.45 & 0.0071&3.40 &0.0003\\
 0.46 & 0.3228 & 0.96 & 0.1685& 1.46 & 0.0721& 1.96 & 0.0250& 2.46 & 0.0069&3.42 &0.0003\\
 0.47 & 0.3192 & 0.97 & 0.1660& 1.47 & 0.0708& 1.97 & 0.0244& 2.47 & 0.0068&3.44 &0.0003\\
 0.48 & 0.3156 & 0.98 & 0.1635& 1.48 & 0.0694& 1.98 & 0.0239& 2.48 & 0.0066&3.46 &0.0003\\
 0.49 & 0.3121 & 0.99 & 0.1611& 1.49 & 0.0681& 1.99 & 0.0233& 2.49 & 0.0064&3.48 &0.0003\\
\hline
\end{tabular}
}

\newpage


\section{Table of critical values for $t$-distributions}
\label{s-disttables-t}

\begin{center}
\begin{tabular}{|l|rrrrrrr|}\hline
& \multicolumn{7}{|c|}{Right-hand tail probability} \\ \cline{2-8}
df	&0.100	&0.050	&0.025	&0.010	&0.005&0.001 & 0.0005\\ \hline
1	&3.078	&6.314	&12.706	&31.821&63.657& 318.309&636.619\\
2	&1.886	&2.920	&4.303	&6.965	&9.925&  22.327& 31.599\\
3	&1.638	&2.353	&3.182	&4.541	&5.841&  10.215& 12.924\\
4	&1.533	&2.132	&2.776	&3.747	&4.604&   7.173&  8.610\\
5	&1.476	&2.015	&2.571	&3.365	&4.032&   5.893&  6.869\\
6	&1.440	&1.943	&2.447	&3.143	&3.707&   5.208&  5.959\\
7	&1.415	&1.895	&2.365	&2.998	&3.499&   4.785&  5.408\\
8	&1.397	&1.860	&2.306	&2.896	&3.355&   4.501&  5.041\\
9	&1.383	&1.833	&2.262	&2.821	&3.250&   4.297&  4.781\\
10	&1.372	&1.812	&2.228	&2.764	&3.169&   4.144&  4.587\\
11	&1.363	&1.796	&2.201	&2.718	&3.106&   4.025&  4.437\\
12	&1.356	&1.782	&2.179	&2.681	&3.055&   3.930&  4.318\\
13	&1.350	&1.771	&2.160	&2.650	&3.012&   3.852&  4.221\\
14	&1.345	&1.761	&2.145	&2.624	&2.977&   3.787&  4.140\\
15	&1.341	&1.753	&2.131	&2.602	&2.947&   3.733&  4.073\\
16	&1.337	&1.746	&2.120	&2.583	&2.921&   3.686&  4.015\\
17	&1.333	&1.740	&2.110	&2.567	&2.898&   3.646&  3.965\\
18	&1.330	&1.734	&2.101	&2.552	&2.878&   3.610&  3.922\\
19	&1.328	&1.729	&2.093	&2.539	&2.861&   3.579&  3.883\\
20	&1.325	&1.725	&2.086	&2.528	&2.845&   3.552&  3.850\\
21	&1.323	&1.721	&2.080	&2.518	&2.831&   3.527&  3.819\\
22	&1.321	&1.717	&2.074	&2.508	&2.819&   3.505&  3.792\\
23	&1.319	&1.714	&2.069	&2.500	&2.807&   3.485&  3.768\\
24	&1.318	&1.711	&2.064	&2.492	&2.797&   3.467&  3.745\\
25	&1.316	&1.708	&2.060	&2.485	&2.787&   3.450&  3.725\\
26	&1.315	&1.706	&2.056	&2.479	&2.779&   3.435&  3.707\\
27	&1.314	&1.703	&2.052	&2.473	&2.771&   3.421&  3.690\\
28	&1.313	&1.701	&2.048	&2.467	&2.763&   3.408&  3.674\\
29	&1.311	&1.699	&2.045	&2.462	&2.756&   3.396&  3.659\\
30	&1.310	&1.697	&2.042	&2.457	&2.750&   3.385&  3.646\\
40	&1.303	&1.684	&2.021	&2.423	&2.704&   3.307&  3.551\\
60	&1.296	&1.671	&2.000	&2.390	&2.660&   3.232&  3.460\\
120	&1.289	&1.658	&1.980	&2.358	&2.617&   3.160&  3.373\\
$\infty$
    &1.282	&1.645	&1.960	&2.326	&2.576&   3.090  & 3.291\\
\hline
\end{tabular}
\end{center}
\emph{Explanation}: For example, the value 3.078 in the top left corner
indicates that for a $t$-distribution with 1 degree of freedom the
probability of values greater than 3.078 is 0.100. The last row shows
critical values for the standard normal distribution.

\newpage


\section{Table of critical values for $\chi^{2}$ distributions}
\label{s_disttables_chi2}



\begin{center}
\begin{tabular}{|l|rrrr|}\hline
& \multicolumn{4}{|c|}{Right-hand Tail probability} \\ \hline
df	&0.100	&0.050	  &0.010&0.001\\ \hline
1	&2.71	&3.84	  &6.63  & 10.828\\
2	&4.61	&5.99	  &9.21  & 13.816\\
3	&6.25	&7.81	  &11.34 & 16.266\\
4	&7.78	&9.49	  &13.28 & 18.467\\
5	&9.24	&11.07	  &15.09 & 20.515\\
6	&10.64	&12.59	  &16.81 & 22.458\\
7	&12.02	&14.07	  &18.48 & 24.322\\
8	&13.36	&15.51	  &20.09 & 26.124\\
9	&14.68	&16.92	  &21.67 & 27.877\\
10	&15.99	&18.31	  &23.21 & 29.588\\
11	&17.28	&19.68	  &24.72 & 31.264\\
12	&18.55	&21.03	  &26.22 & 32.909\\
13	&19.81	&22.36	  &27.69 & 34.528\\
14	&21.06	&23.68	  &29.14 & 36.123\\
15	&22.31	&25.00	  &30.58 & 37.697\\
16	&23.54	&26.30	  &32.00 & 39.252\\
17	&24.77	&27.59	  &33.41 & 40.790\\
18	&25.99	&28.87	  &34.81 & 42.312\\
19	&27.20	&30.14	  &36.19 & 43.820\\
20	&28.41	&31.41	  &37.57 & 45.315\\
25	&34.38	&37.65	  &44.31 & 52.620\\
30	&40.26	&43.77	  &50.89 & 59.703\\
40	&51.81	&55.76	  &63.69 & 73.402\\
50	&63.17	&67.50	  &76.15 & 86.661\\
60	&74.40	&79.08	  &88.38 & 99.607\\
70	&85.53	&90.53	  &100.43&112.317\\
80	&96.58	&101.88	  &112.33&124.839\\
90	&107.57	&113.15   &124.12&137.208\\
100	&118.50	&124.34	  &135.81&149.449\\
\hline
\end{tabular}
\end{center}
\emph{Explanation}: For example, the value 2.71 in the top left corner
indicates that for a $\chi^{2}$ distribution with 1 degree of freedom
the probability of values greater than 2.71 is 0.100.
