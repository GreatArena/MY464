\chapter{Analysis of 3-way contingency tables}
\label{c_3waytables}

In Section \ref{s_descr1_2cat} and Chapter \ref{c_tables} we discussed
the analysis of two-way contingency tables (crosstabulations) for
examining the associations between two categorical variables. In this
section we extend this by introducing the basic ideas of
\textbf{multiway contingency tables} which include more than two
categorical variables. We focus solely on the simplest instance of them,
a \textbf{three-way table} of three variables.

This topic is thematically related also to some of Chapter
\ref{c_regression}, in that a multiway contingency table can be seen as
a way of implementing for categorical variables
the ideas of statistical control
that were also a feature
of the multiple linear regression model of Section
\ref{s_regression_multiple}. Here, however, we will not consider formal
regression models for categorical variables (these are mentioned only
briefly at the end of the chapter). Instead, we give examples of
analyses which simply apply familiar methods for two-way tables repeatedly
for tables of two variables at fixed values of a third variable.

The discussion is organised arond three examples. In each case we start
with a two-way table, and then introduce a third variable which we want
to control for. This reveals various features in the examples, to
illustrate the types of findings that may be uncovered by statistical
control.

\textbf{Example 9.1: Berkeley admissions}

Table \ref{t_berkeley1} summarises data on applications for admission to
graduate study at the University of California, Berkeley, for the fall
quarter 1973\footnote{These data, which were produced by the Graduate
Division of UC Berkeley, were first discussed in Bickel, P.\ J., Hammel,
E.\ A., and O'Connell, J.\ W. (1975), ``Sex bias in graduate admissions:
Data from Berkeley'', \emph{Science} \textbf{187}, 398--404. They have
since become a much-used teaching example. The version of the data
considered here are from Freedman, D., Pisani, R., and Purves, R.,
\emph{Statistics} (W.\ W.\ Norton, 1978).}. The data are for five of the
six departments with the largest number of applications, labelled below
Departments 2--5 (Department 1 will be discussed at the end of this
section). Table \ref{t_berkeley1} shows the two-way contingency table of
the sex of the applicant and whether he or she was admitted to the
university.

\begin{table}
\caption{Table of sex of applicant vs.\ admission in the Berkeley
admissions data. The
column labelled `\% Yes' is the percentage of applicants admitted within
each row.}
\label{t_berkeley1}
\begin{center}
\begin{tabular}{|l|rr|r|r|}\hline
& \multicolumn{2}{|c|}{Admitted} & & \\
Sex & No & Yes & \% Yes& Total \\ \hline
Male & 1180 &  686 & 36.8 & 1866 \\
Female & 1259 & 468 & 27.1 & 1727 \\ \hline
Total & 2439 & 1154 & 32.1 & 3593 \\
\hline
\multicolumn{5}{l}{{\small $\chi^{2}=38.4$, $df=1$, $P<0.001$}}
\end{tabular}
\end{center}
\vspace*{-2ex}
\end{table}

The percentages in Table \ref{t_berkeley1} show that men were more
likely to be admitted, with a 36.8\% success rate compared to 27.1\% for
women. The difference is strongly significant, with $P<0.001$ for the
$\chi^{2}$ test of independence. If this association was interpreted
causally, it might be regarded as evidence of sex bias in the admissions
process. However, other important variables may also need to be
considered in the analysis. One of them is the academic department
to which an applicant had applied. Information on the department as well
as sex and admission is shown in Table \ref{t_berkeley2}.

\begin{table}[t]
\caption{
Sex of applicant vs.\ admission by academic department
in the Berkeley admissions data.
}
\label{t_berkeley2}
\begin{center}
\begin{tabular}{|ll|rr|r|r|}\hline
& & \multicolumn{2}{|c|}{Admitted} & &  \\
Department & Sex & No & Yes & \% Yes& Total  \\ \hline
2& Male & 207 &  353 & 63.0 & 560 \\
& Female & 8 & 17 & 68.0 & 25 \\ \hline
& Total & 215 & 370 & 63.2 & 585 \\ \hline
\multicolumn{2}{|l}{$\,$ }&  \multicolumn{4}{l|}{$\chi^{2}=0.25$,
$P=0.61$}\\
\hline
3& Male & 205 &  120 & 36.9 & 325 \\
& Female & 391 & 202 & 34.1 & 593 \\ \hline
& Total & 596 & 322 & 35.1 & 918 \\ \hline
\multicolumn{2}{|l}{$\,$ }&
\multicolumn{4}{l|}{$\chi^{2}=0.75$, $P=0.39$}\\
\hline
4& Male & 279 &  138 & 33.1 & 417 \\
& Female & 244 & 131 & 34.9 & 375 \\ \hline
& Total & 523 & 269 & 34.0 & 792 \\ \hline
\multicolumn{2}{|l}{$\,$ }&
\multicolumn{4}{l|}{$\chi^{2}=0.30$, $P=0.59$}\\
\hline
5& Male & 138 &  53 & 27.7 & 191 \\
& Female & 299 & 94 & 23.9 & 393 \\ \hline
& Total & 437 & 147 & 25.2 & 584 \\ \hline
\multicolumn{2}{|l}{$\,$ }&
\multicolumn{4}{l|}{$\chi^{2}=1.00$, $P=0.32$}\\
\hline
6& Male & 351 & 22 & 5.9 & 373 \\
& Female & 317 & 24 & 7.0 & 341 \\ \hline
& Total & 668 & 46 & 6.4 & 714 \\ \hline
\multicolumn{2}{|l}{$\,$ }&
\multicolumn{4}{l|}{$\chi^{2}=0.38$, $P=0.54$}\\
\hline
Total & & 2439 & 1154 & 32.1 & 3593 \\ \hline
\end{tabular}
\end{center}
\end{table}

Table \ref{t_berkeley2} is a \emph{three-way} contingency table, because
each of its internal cells shows the number of applicants with a
particular combination of three variables: department, sex and admission
status. For example, the frequency 207 in the top left corner indicates
that there were 207 male applicants to department 2 who were not
admitted.
Table
\ref{t_berkeley2} is presented in the form of a series of tables of sex
vs.\ admission, just like in the original two-way table
\ref{t_berkeley1}, but now with one table for each department. These are
known as \textbf{partial tables} of sex vs.\ admission,
\textbf{controlling for} department. The word ``control'' is used here
in the same sense as before: each partial table
summarises the data for the applicants to a single department, so the
variable ``department'' is literally held constant within the partial
tables.

Table \ref{t_berkeley2} also contains the marginal distributions of sex and admission status within each
department. They can be used to construct the other two possible two-way
tables for these variables, for department vs.\ sex of applicant and
department vs.\ admission status. This information, summarised in Table
\ref{t_berkeley3}, is discussed further below.

The association between sex and admission within each partial table can
be examined using methods for two-way tables. For every one of
them, the $\chi^{2}$ test shows that the hypothesis of independence
cannot be rejected, so there is no evidence of sex bias within any
department. The apparent association in Table
\ref{t_berkeley1} is thus spurious, and disappears when we control for
department. Why this happens can be understood by considering the
distributions of sex and admissions across departments, as shown in
Table \ref{t_berkeley3}. Department is clearly associated with sex of
the applicant: for example, almost all of the applicants to department
2, but only a third of the applicants to department 5 are men.
Similarly, there is an association between department and admission: for
example, nearly two thirds of the applicants to department 2 but only a
quarter of the applicants to department 5 were admitted. It is the
combination of these associations which induces the spurious association
between sex and admission in Table \ref{t_berkeley1}. In essence, women
had a lower admission rate overall because relatively more of them
applied to the more selective departments and fewer to the
easy ones.

\begin{table}
\caption{Percentages of male applicants and applicants admitted by
department in the Berkeley admissions data.}
\label{t_berkeley3}
\begin{center}
\begin{tabular}{|l|rrrrr|}\hline
& \multicolumn{5}{|c|}{Department}\\
Of all applicants & 2 & 3 & 4 & 5 & 6 \\ \hline
\% Male & 96 & 35 & 53 & 33 & 52 \\
\% Admitted & 63 & 35 & 34 & 25 & 6 \\
\hline
Number of applicants & 585 & 918 & 792 & 584 & 714 \\
\hline
\end{tabular}
\end{center}
\vspace*{-2ex}
\end{table}

One possible set of causal connections leading to a spurious association
between $X$ and $Y$ was represented graphically by Figure
\ref{f_xyzspurious}. There are, however, other possibilities which may
be more appropriate in particular cases. In the admissions example,
department (corresponding to the control variable $Z$) cannot be
regarded as the cause of the sex of the applicant. Instead, we may
consider the causal chain Sex $\longrightarrow$ Department
$\longrightarrow$ Admission. Here department is an \emph{intervening
variable} between sex and admission rather than a common cause of them.
We can still argue that sex has an effect on admission, but it is an
\emph{indirect effect} operating through the effect of sex on choice of
department. The distinction is important for the original research question
behind these data, that of possible sex bias in admissions.
A direct effect of sex on likelihood on admission might be evidence of
such bias, because it might indicate that departments are treating male
and female candidates differently. An indirect effect of the kind found
here does not suggest bias, because it results from the applicants' own
choices of which department to apply to.

In the admissions example a strong association in the initial two-way
table was ``explained away'' when we controlled for a third variable.
The next example is one where controlling leaves the initial association
unchanged.

\textbf{Example 9.2:}
\textbf{Importance of short-term gains
for investors (continued)}

Table \ref{t_investors} on page \pageref{t_investors} showed a
relatively strong association between a person's age group and his or
her attitude towards short-term gains as an investment goal. This
association is also strongly significant, with $P<0.001$ for the
$\chi^{2}$ test of independence.
Table \ref{t_investors3}
shows the crosstabulations of these variables, now controlling also for the
respondent's sex. The association is now still significant in both
partial tables. An investigation of the row proportions suggests that
the pattern of association is very similar in both tables, as is its
strength as measured by the $\gamma$ statistic ($\gamma=-0.376$ among
men and $\gamma=-0.395$ among women). The conclusions obtained from the
original two-way table are thus unchanged after controlling for sex.

\begin{table}[t]
\caption{
Frequencies of respondents by age group and attitude towards short-term
gains in Example 9.2, controlling for sex of respondent.
The numbers below the frequencies are proportions within rows.}
\label{t_investors3}
\begin{center}
\begin{tabular}{|l|rrrr|r|}\hline
\textbf{MEN}& \multicolumn{4}{|c|}{Importance of short-term gains } & \\
 & & Slightly & & Very & \\
Age group & Irrelevant & important & Important & important & Total \\ \hline
Under 45 &  29 &  35 &  30 &  22 & 116 \\
& 0.250 & 0.302 & 0.259 & 0.190 & 1.000 \\
45--54 &  83 &  60 &  52 &  29 & 224 \\
& 0.371 & 0.268 & 0.232 & 0.129 & 1.000 \\
55--64 & 116 &  40 &  28 &  16 & 200 \\
& 0.580 & 0.200 & 0.140 & 0.080 & 1.000 \\
65 and over &  150 &  53 &  16 &  12 & 231 \\
& 0.649 & 0.229 & 0.069 & 0.052 & 1.000 \\
\hline
Total & 378 & 188 & 126 & 79 & 771 \\
& 0.490 & 0.244 & 0.163 & 0.102 & 1.000 \\
\hline
\multicolumn{6}{l}{$\chi^{2}=82.4$, $df=9$, $P<0.001$.
$\gamma=-0.376$}
\end{tabular}
\end{center}

\begin{center}
\begin{tabular}{|l|rrrr|r|}\hline
\textbf{WOMEN}& \multicolumn{4}{|c|}{Importance of short-term gains } & \\
 & & Slightly & & Very & \\
Age group & Irrelevant & important & Important & important & Total \\ \hline
Under 45 &  8 &  10 &  8 &  4 & 30 \\
& 0.267 & 0.333 & 0.267 & 0.133 & 1.000 \\
45--54 &  28 &  17 &  5 &  8 & 58 \\
& 0.483 & 0.293 & 0.086 & 0.138 & 1.000 \\
55--64 & 37 &  9 &  3 &  4 & 53 \\
& 0.698 & 0.170 & 0.057 & 0.075 & 1.000 \\
65 and over &  43 &  11 &  3 &  3 & 60 \\
& 0.717 & 0.183 & 0.050 & 0.050 & 1.000 \\
\hline
Total & 116 & 47 & 19 & 19 & 201 \\
& 0.577 & 0.234 & 0.095 & 0.095 & 1.000 \\
\hline
\multicolumn{6}{l}{$\chi^{2}=27.6$, $df=9$, $P=0.001$.
$\gamma=-0.395$}
\end{tabular}
\end{center}
\end{table}

\textbf{Example 9.3}: \textbf{\emph{The Titanic}}

The passenger liner RMS \emph{Titanic} hit an iceberg and sank in the
North Atlantic on 14 April 1912, with heavy loss of life. Table
\ref{t_titanic2} shows a crosstabulation of the people on board the
\emph{Titanic}, classified according to their status (as male passenger,
female or child passenger, or member of the ship's crew) and whether
they survived the sinking\footnote{The data are from the 1912 report of
the official British Wreck Commissioner's inquiry into the sinking,
available at \texttt{www.titanicinquiry.org}.}. The $\chi^{2}$ test of
independence has $P<0.001$ for this table, so there are statistically
significant differences in probabilities of survival between the groups.
The table suggests, in particular,
that women and children among the passengers were more likely to survive than male
passengers or the ship's crew.

\begin{table}
\caption{Survival status of the people aboard the \emph{Titanic},
divided into three groups. The numbers in brackets are
proportions of survivors and non-survivors within each group.}
\label{t_titanic2}
\begin{center}
\begin{tabular}{|l|rr|r|}\hline
& \multicolumn{2}{|c|}{Survivor} & \\
Group & Yes & No & Total \\ \hline
Male passenger & 146 & 659 & 805 \\
 & (0.181) & (0.819) & (1.000) \\
Female or child passenger & 353 & 158 & 511 \\
 & (0.691) & (0.309) & (1.000) \\
Crew member & 212 & 673 & 885 \\
 & (0.240) & (0.760) & (1.000) \\ \hline
Total & 711 & 1490 & 2201 \\
 & (0.323) & (0.677) & (1.000) \\
\hline
\multicolumn{4}{l}{\small $\chi^{2}=418$, $\text{df}=2$, $P<0.001$.}
\end{tabular}
\end{center}
\end{table}

We next control also for the
class in which a person was travelling, classified as first, second or
third class. Since class does not apply to the ship's crew, this
analysis is limited to the passengers, classified as men vs.\ women and
children. The two-way table of sex by survival status for them is
given by Table \ref{t_titanic2}, ignoring the row for crew members. This
association is strongly significant, with
$\chi^{2}=344$ and $P<0.001$.


\begin{table}
\caption{
%Survival status of the people aboard the \emph{Titanic},
%divided into three groups (Example 6.2). The numbers in brackets are
%proportions within each row (i.e.\ within-group proportions of survivors
%and non-survivors).
Survival status of the passengers of the \emph{Titanic}, classified by
class and sex.
The numbers below the frequencies are proportions within rows.
}
\label{t_titanic3}
\begin{center}
\begin{tabular}{|l|l|rr|r|}\hline
& & \multicolumn{2}{|c|}{Survivor} & \\
Class & Group & Yes & No & Total \\
\hline
\multicolumn{5}{l}{\vspace*{-2ex}}
\\ \hline
First & Man & 57 & 118 & 175 \\
&  & 0.326 & 0.674 & 1.000 \\
& Woman or child & 146 & 4 & 150 \\
&  & 0.973 & 0.027 & 1.000 \\ \hline
& Total & 203 & 122 & 325 \\
&  & 0.625 & 0.375 & 1.000 \\
\hline
\multicolumn{5}{l}{\vspace*{-2ex}}
\\ \hline
Second & Man & 14 & 154 & 168 \\
&  & 0.083 & 0.917 & 1.000 \\
& Woman or child & 104 & 13 & 117 \\
&  & 0.889 & 0.111 & 1.000 \\ \hline
& Total & 118 & 167 & 285 \\
&  & 0.414 & 0.586 & 1.000 \\
\hline \multicolumn{5}{l}{\vspace*{-2ex}}\\ \hline
Third & Man & 75 & 387 & 462 \\
&  & 0.162 & 0.838 & 1.000 \\
& Woman or child & 103 & 141 & 244 \\
&  & 0.422 & 0.578 & 1.000 \\ \hline
& Total & 178 & 528 & 706 \\
&  & 0.252 & 0.748 & 1.000 \\
\hline \multicolumn{5}{l}{\vspace*{-2ex}}\\ \hline
Total & & 499 & 817 & 1316 \\
&  & 0.379 & 0.621 & 1.000 \\
\hline
\end{tabular}
\end{center}
\end{table}

Two-way tables involving class (not shown here)
suggest that it is mildly associated with sex (with percentages of
men 54\%, 59\% and 65\% in first, second and third class respectively)
and strongly associated with survival (with 63\%, 41\% and 25\% of the
passengers surviving). It
is thus possible that class might influence the association between sex and
survival. This is investigated in Table \ref{t_titanic3}, which shows
the partial associations between sex and survival status, controlling
for class. This association is strongly significant (with $P<0.001$ for
the $\chi^{2}$ test) in every partial table, so it is clearly not
explained away by associations involving class. The direction of the
association is also the same in each table, with women and children more
likely to survive than men among passengers of every class.

The presence and direction of the association in the two-way Table
\ref{t_titanic2} are thus preserved and similar in every partial table
controlling for class. However, there appear to be differences in the
\emph{strength} of the association between the partial tables. Considering, for example, the ratios of the proportions in
each class, women and children were about 3.0 ($=0.973/0.326$) times
more likely to survive than men in first class, while the ratio was
about 10.7 in second class and 2.6 in the third. The contrast of
men vs.\ women and children was thus strongest among
second-class passengers. This example differs in this respect from the
previous ones, where the associations were similar in each partial
table, either because they were all essentially zero (Example 9.1) or
non-zero but similar in both direction and strength (Example 9.2).

We are now considering three variables, class, sex and survival.
Although it is not necessary for this analysis to divide them into
explanatory and response variables, introducing such a distinction is
useful for discussion of the results. Here it is most natural to treat
survival as the response variable, and both class and sex as explanatory
variables for survival. The associations in the partial tables in Table
\ref{t_titanic3} are then partial associations between
the response variable and one of the explanatory variables (sex),
controlling for the other explanatory variable (class). As discussed
above, the strength of this partial association is different for
different values of class. This is an example of a statistical
\textbf{interaction}. In general, there is an interaction between two
explanatory variables if the strength and nature of the partial
association of (either) one of them on a response variable depends on
the value at which the other explanatory variable is controlled. Here
there is an interaction between class and sex, because the association
between sex and survival is different at different levels of class.

Interactions are an important but challenging element of many
statistical analyses. Important, because they often correspond to
interesting and subtle features of associations in the data.
Challenging, because understanding and interpreting them involves
talking about (at least) three variables at once. This can seem rather
complicated, at least initially. It adds to the difficulty that
interactions can take many forms. In the \emph{Titanic} example, for
instance, the nature of the class-sex interaction was that the
association between sex and survival was in the same direction but of
different strengths at different levels of class. In other cases
associations may disappear in some but not all of the partial tables, or
remain strong but in different directions in different ones. They may
even all or nearly all be in a different direction from the association
in the original two-way table, as in the next example.



\textbf{Example 9.4: Smoking and mortality}

A health survey was carried out in Whickham near Newcastle upon Tyne in
1972--74, and a follow-up survey of the same respondents twenty years
later\footnote{The two studies are reported in Tunbridge, W.\ M.\ G.\ et
al.\ (1977). ``The spectrum of thyroid disease in a community: The
Whickham survey''.
\emph{Clinical Endocrinology} \textbf{7}, 481--493,
and Vanderpump, M.\ P.\ J.\ et al.\ (1995). ``The incidence of thyroid
disorders in the community: A twenty-year follow-up of the Whickham
survey''.
\emph{Clinical Endocrinology} \textbf{43}, 55--69.
The data are used to illustrate Simpson's paradox by Appleton, D.\ R.\
et al.\ (1996). ``Ignoring a covariate: An example of Simpson's
paradox''. \emph{The American Statistician} \textbf{50}, 340--341.
}. Here we consider only the $n=1314$ female respondents who were
classified by the first survey either as current smokers or as never
having smoked. Table \ref{t_whickham1} shows the crossclassification of
these women according to their smoking status in 1972--74 and whether
they were still alive twenty years later. The $\chi^{2}$ test indicates
a strongly significant association (with $P=0.003$), and the numbers
suggest that a smaller proportion of smokers than of nonsmokers had died
between the surveys. Should we thus conclude that smoking helps to keep
you alive? Probably not, given that it is known beyond reasonable
doubt that the causal relationship between smoking and mortality
is in the opposite direction. Clearly the picture has been distorted by
failure to control for some relevant further variables. One such
variable is the age of the respondents.

\begin{table}[t]
\caption{Table of smoking status in 1972--74 vs.\ twenty-year survival
among the respondents in Example 9.4. The numbers below the frequencies
are proportions within rows.}
\label{t_whickham1}
\begin{center}
\begin{tabular}{|l|rr|r|}\hline
Smoker & Dead & Alive & Total \\ \hline
Yes & 139 & 443 & 582  \\
& 0.239 & 0.761 & 1.000 \\
No & 230 & 502 & 732 \\
& 0.314 & 0.686 & 1.000 \\ \hline
Total & 369 & 945 & 1314 \\
& 0.281 & 0.719 & 1.000  \\
\hline
\end{tabular}
\end{center}
\end{table}

Table \ref{t_whickham2} shows the partial tables of smoking vs.\
survival controlling for age at the time of the first survey, classified
into seven categories. Note first that this three-way table appears
somewhat different from those shown in Tables \ref{t_berkeley2},
\ref{t_investors3} and \ref{t_titanic3}. This is because one variable,
survival status, is summarised only by the percentage of survivors
within each combination of age group and smoking status. This is a
common trick to save space in three-way tables involving
dichotomous variables like survival here. The full table can easily be
constructed from these numbers if needed. For example, 98.4\% of the
nonsmokers aged 18--24 were alive at the time of the second survey.
Since there were a total of 62 respondents in this group (as shown in
the last column), this means that 61 of them (i.e.\ 98.4\%) were alive
and 1 (or 1.6\%) was not.

The percentages in Table \ref{t_whickham2} show that in five of the
seven age groups the proportion of survivors is higher among nonsmokers
than smokers, i.e.\ these partial associations in the sample are in the
opposite direction from the association in Table \ref{t_whickham1}. This
reversal is known as \textbf{Simpson's paradox}. The term is somewhat
misleading, as the finding is not really paradoxical in any logical
sense. Instead, it is again a consequence of a particular pattern of
associations between the control variable and the other two variables.

\begin{table}[t]
\caption{Percentage of respondents in Example 9.4
surviving at the time of the second survey, by
smoking status and age group in 1972--74.}
\label{t_whickham2}
\begin{center}
\begin{tabular}{|l|rr|rr|}\hline
& \multicolumn{2}{|c|}{\% Alive after 20 years} &
 \multicolumn{2}{|c|}{Number (in 1972--74)} \\
Age group & Smokers & Nonsmokers & Smokers & Nonsmokers \\ \hline
18--24 & 96.4 & 98.4 & 55 & 62 \\
25--34 & 97.6 & 96.8 & 124 & 157 \\
35--44 & 87.2 & 94.2 & 109 & 121 \\
45--54 & 79.2 & 84.6 & 130 & 78 \\
55--64 & 55.7 & 66.9 & 115 & 121 \\
65--74 & 19.4 & 21.7 & 36 & 129 \\
75-- & 0.0 & 0.0 & 12 & 64 \\
\hline
All age groups & 76.1 & 68.6 & 582 & 732 \\ \hline
\end{tabular}
\end{center}
\vspace*{-2ex}
\end{table}

The two-way tables of age by survival and age by smoking are shown side
by side in Table \ref{t_whickham3}. The table is somewhat elaborate and
unconventional, so it requires some explanation. The rows of the table
correspond to the age groups, identified by the second column, and the
frequencies of respondents in each age group are given in the last
column. The left-hand column shows the percentages of survivors within
each age group. The right-hand side of the table gives the two-way table
of age group and smoking status. It contains percentages calculated both
within the rows (without parentheses) and columns (in parentheses) of
the table. As an example, consider numbers for the age group 18--24.
There were 117 respondents in this age group at the time of the first
survey. Of them, 47\% were then smokers and 53\% were nonsmokers, and
97\% were still alive at the time of the second survey. Furthermore,
10\% of all the 582 smokers, 9\% of all the 732 nonsmokers and 9\% of
the 1314 respondents overall were in this age group.

\begin{table}[t]
\caption{Two-way contingency tables of age group vs.\ survival (on the
left) and age group vs.\ smoking (on the right)
in Example 6.4. The percentages in parentheses are column percentages
(within smokers or nonsmokers) and the ones without parentheses are row
percentages (within age groups).
}
\label{t_whickham3}
\begin{center}
\begin{tabular}{|r||l||cc|c||r|}\hline
& & \multicolumn{2}{c}{Row and column \%} &
\multicolumn{2}{|c|}{Total} \\
\% Alive & Age group & Smokers & Nonsmokers &
\multicolumn{1}{c}{\%} & Count \\ \hline
97 & 18--24 & 47 & 53 & 100 & 117 \\
& & (10) & (9) & (9) & \\
97 & 25--34 &  44 & 56 & 100 & 281 \\
& & (21) & (21) & (21) & \\
91 & 35--44 & 47 & 53 & 100 & 230 \\
& & (19) & (17) & (18) & \\
81 & 45--54 & 63 & 38 & 100 & 208 \\
& & (22) & (11) & (16) & \\
61 & 55--64 & 49 & 51 & 100 & 236 \\
& & (20) & (17) & (13) & \\
21 & 65--74 & 22 & 78 & 100 & 165 \\
& & (6) & (18) & (13) & \\
0 & 75-- & 17 & 83 & 100 & 77 \\
& & (2) & (9) & (6) & \\
\hline
72 & Total \% & 44 & 56 & 100 & \\
& & (100) & (100) & (100) & \\ \hline
945 & Total count & 582 & 732 & & 1314 \\
\hline
\end{tabular}
\end{center}
\end{table}

Table \ref{t_whickham3} shows a clear association between age and
survival, for understandable reasons mostly unconnected with smoking.
The youngest respondents of the first survey were highly likely and the
oldest unlikely to be alive twenty years later. There is also an
association between age and smoking: in particular, the proportion of
smokers was lowest among the oldest respondents. The implications of
this are seen perhaps more clearly by considering the column
proportions, i.e.\ the age distributions of smokers and nonsmokers in
the original sample. These show that the group of nonsmokers was
substantially older than that of the smokers; for example, 27\% of the
nonsmokers but only 8\% of the smokers belonged to the two oldest age
groups. It is this imbalance which explains why nonsmokers, more of whom
are old, appear to have lower chances of survival until we control for
age.

The discussion above refers to the \emph{sample} associations between smoking
and survival in the partial tables, which suggest that mortality is
higher among smokers than nonsmokers. In terms of statistical
significance, however, the evidence is fairly weak: the association is
even borderline significant only in the 35--44 and 55--64 age groups,
with $P$-values of 0.063 and 0.075 respectively for the $\chi^{2}$ test.
This is an indication not so much of lack of a real association but of
the fact that these data do not provide much power for detecting it.
Overall twenty-year mortality is a fairly rough measure of health status
for comparisons of smokers and nonsmokers, especially in the youngest
and oldest age groups where mortality is either very low or very high
for everyone. Differences are likely to be have been further diluted by
many of the original smokers having stopped smoking between the surveys.
This example should thus not be regarded as a serious examination of the
health effects of smoking, for which much more specific data and more
careful analyses are required\footnote{For one remarkable example of
such studies, see Doll, R.\ et al.\ (2004), ``Mortality in relation to
smoking: 50 years' observations on male British doctors'', \emph{British
Medical Journal} \textbf{328}, 1519--1528, and Doll, R.\ and Hill, A.\
B.\ (1954), ``The Mortality of doctors in relation to their smoking
habits: A preliminary report'', \emph{British Medical Journal}
\textbf{228}, 1451--1455. The older paper is reprinted
together with the more recent one
in the 2004 issue
of \emph{BMJ}.}.

The Berkeley admissions data discussed earlier provide another example of
a partial Simpson's paradox. Previously we considered only departments
2--6, for none of which there was a significant partial association
between sex and admission. For department 1, the partial table indicates
a strongly significant difference in favour of women, with 82\% of the
108 female applicants admitted, compared to 62\% of the 825 male
applicants. However, the two-way association between sex and admission
for departments 1--6 combined remains strongly significant and shows an
even larger difference in favour of men than before. This result can now
be readily explained as a result of imbalances in sex ratios and rates
of admission between departments. Department 1 is both easy to get into
(with 64\% admitted) and heavily favoured by men (88\% of the
applicants). These features combine to contribute to higher admissions
percentages for men overall, even though within department 1 itself
women are more likely to be admitted.

In summary, the examples discussed above demonstrate that many things
can happen to an association between two variables when we control for a
third one. The association may disappear, indicating that it was
spurious, or it may remain similar and unchanged in all of the partial
tables. It may also become different in different partial tables,
indicating an interaction. Which of these occurs depends on the patterns
of associations between the control variable and the other two
variables. The crucial point is that the two-way table alone cannot
reveal which of these cases we are dealing with, because the counts in
the two-way table could split into three-way tables in many different
ways. The only way to determine how controlling for other variables will
affect an association is to actually do so. This is the case not only
for multiway contingency tables, but for all methods of statistical
control, in particular multiple linear regression and other regression models.

Two final remarks round off our discussion of
multiway contingency tables:
\begin{itemize}
\item
Extension of the ideas of three-way tables to four-way and larger
contingency tables is obvious and straightforward. In such tables, every
cell corresponds to the subjects with a particular combination of the
values of four or more variables. This involves no new conceptual
difficulties, and the only challenge is how to arrange the table for
convenient presentation. When the main interest is in associations
between a particular pair of two variables, the usual solution is to
present a set of partial two-way tables for them, one for each
combination of the other (control) variables. Suppose, for instance,
that in the university admissions example we had obtained similar data
at two different years, say 1973 and 2003. We would then have four
variables: year, department, sex and admission status. These could be
summarised as in Table \ref{t_berkeley2}, except that each partial table
for sex vs.\ admission would now be conditional on the values of both
year and department.
The full four-way table
would then consist of ten $2\times 2$ partial tables, one for each of
the ten combinations of two years and five
departments, (i.e.\ applicants to Department 2 in 1973, Department 2 in
2003, and so on to Department 6 in 2003).
\item
The only inferential tool for multiway contingency tables discussed here
was to arrange the table as a set of two-way partial tables and to
apply the $\chi^{2}$ test of independence to each of them. This is a
perfectly sensible approach and a great improvement over just analysing
two-way tables. There are, however, questions which cannot easily be
answered with this method. For example, when can we say that
associations in different partial tables are different enough for us to
declare that there is evidence of an interaction? Or what if we want to
consider many different partial associations, either for a response
variable with each of the other variables in turn, or because there is
no single response variable? More powerful methods are
required for such analyses. They are multiple regression models like the
multiple linear regression of Section \ref{s_regression_multiple}, but
modified so that they become approriate for categorical response variables. Some of these
models are introduced on the course MY452.
\end{itemize}
