\chapter{Introduction}
\label{c_intro}

\vspace*{-2.5ex}
\section{What is the purpose of this course?}
\label{s_intro_purpose}

%To begin our story with the beginning of our story,
%let us consider
%the title of this course:
%\vspace*{-2ex}
The title of any course should be descriptive of its contents.
This one is called
%The title of this course is
\begin{center}
%\underline{MI451: Quantitative Analysis I: Description and Inference}
\textbf{MY451: Introduction to Quantitative Analysis}
\end{center}
Every part of this tells us something about the nature of the
course:

The \textbf{M} stands for \emph{Methodology} of social research. Here
\emph{research} refers to activities aimed at obtaining new knowledge
about the world, in the case of the social sciences the \emph{social}
world of people and their institutions and interactions. Here we are
concerned solely with \emph{empirical} research, where such knowledge is
based on information obtained by \emph{observing} what goes on in that
world. There are many different ways (\emph{methods}) of making such
observations, some better than others for deriving valid knowledge.
``Methodology'' refers both to the methods used in particular studies,
and the study of research methods in general.

The word \textbf{analysis} indicates the area of research methodology
that the course is about. In general, any empirical research project
will involve at least the following stages:
\begin{enumerate}
\item
Identifying a research \emph{topic}
\item
Formulating \emph{research questions}
\item
Deciding what kinds of \emph{information} to collect to try to answer the
research questions, and deciding how to collect it and where to collect
it from
\item
Collecting the information
\item
\emph{Analysing} the information in appropriate ways to answer the
research questions
\item
\emph{Reporting} the findings
\end{enumerate}

The empirical information collected in the research process is often
referred to as \emph{data}. This course is mostly about
some basic methods for step 5, the \emph{analysis} of such data.

Methods of analysis, however competently used, will not be very useful
unless other parts of the research process have also been carried out
well. These other parts, which (especially steps 2--4 above) can be
broadly termed \emph{research design}, are covered on other courses,
such as MY400 (Fundamentals of Social Science Research Design) or comparable
courses at your own department. Here we will mostly not consider
research design, in effect assuming that we start at a point where we
want to analyse some data which have been collected in a sensible
way to answer meaningful research questions. However, you should bear in
mind throughout the course that in a real research situation both good
design and good analysis are essential for success.

The word \textbf{quantitative} in the title of the course indicates that
the methods you will learn here are used to analyse quantitative data. This
means that the data will enter the analysis in the form of
\emph{numbers} of some kind. In social sciences, for example,
data obtained from administrative records or from surveys using
structured interviews are typically quantitative. An alternative is
\emph{qualitative} data, which are not rendered into numbers for the
analysis. For example, unstructured interviews, focus groups and
ethnography typically produce mostly qualitative data. Both quantitative
and qualitative data are important and widely used in social research.
For some research questions, one or the other may be clearly more
appropriate, but in many if not most cases the research would benefit
from collecting both qualitative and quantitative data. This course will
concentrate solely on quantitative data analysis, while the collection
and analysis of qualitative data are covered on other courses (e.g.\
MY421, MY426 and MY427), which we hope you will also be taking.

All the methods taught here, and almost all approaches used for
quantitative data analysis in the social sciences in general, are
\emph{statistical} methods. The defining feature of such methods is that
randomness and probability play an essential role in them; some of the
ways in which they do so will become apparent later, others need not
concern us here. The title of the course could thus also have included
the word \emph{statistics}. However, the Department of Methodology courses on
statistical methods (e.g.\ MY451, MY465, MY452, MY455 and MY459) have traditionally been
labelled as courses on ``quantitative analysis'' rather than
``statistics''. This is done to indicate that they differ from classical introductory
statistics courses in some ways, especially in the presentation being
less mathematical.

The course is called an ``\textbf{Introduction} to Quantitative Analysis''
because it is an introductory course which does not assume
that you have learned any statistics before. MY451 or a comparable
course should be taken before more advanced
courses on quantitative methods. Statistics is a cumulative
subject where later courses build on material learned on earlier ones.
Because MY451 is introductory, it will start with very simple methods,
and many of the more advanced (and powerful) ones will only be covered
on the later courses. This does not, however, mean that you are wasting
your time here even if it is methods from, say, MY452 that you will
eventually need most: understanding the material of this course is
essential for learning more advanced methods.

%The words \textbf{description} and \textbf{inference} in the title refer
%to the two kinds of questions the methods will be used to answer. This
%is explained further in Section \ref{ss_intro_def_descr} below.

Finally, the course has an \textbf{MY} code, rather than GV, MC, PS, SO,
SP, or whatever is the code of your own department. MY451
is taken by students from many different degrees and departments, and
thus cannot be tailored to any one of them specifically. For example, we
will use examples from many different social sciences. However, this
generality is definitely a good thing: the reason we \emph{can} teach
all of you together is that statistical methods (just like the
principles of research design or qualitative research) are generic and
applicable to the analysis of quantitative data in all fields of social
research. There is not, apart from differences in emphases and
priorities, one kind of statistics for sociology and another for political
science or economics, but one coherent set of principles and methods for
all of them (as well as for psychiatry, epidemiology, biology,
astrophysics and so on). After this course you will have taken the first
steps in learning about all of that.

At the end of the course you should be familiar with certain methods of
statistical analysis. This will enable you to be both a user and a
consumer of statistics:
\begin{itemize}
\item You will be able to use
the methods to analyse your own data and to report the results of the
analyses.
\item
Perhaps even more importantly, you will also be able to understand
(and possibly criticize) their use in other people's research. Because
interpreting results is typically somewhat easier than carrying out new
analyses, and because all statistical methods use the same basic ideas
introduced here, you will even have some understanding of many of the
techniques not discussed on this course.
\end{itemize}
Another pair of different but complementary aims of the
course is that MY451 is both a self-contained unit and a prerequisite for
courses that follow it:
\begin{itemize}
\item
If this is the last statistics course you will take, it will enable you
to understand and use the particular methods covered here. This includes
the technique of linear regression modelling (described in Chapter
\ref{c_regression}), which is arguably the most important and commonly
used statistical method of all. This course can, however, introduce only
the most important elements of linear regression, while some of the more
advanced ones are discussed only on MY452.
\item
The ideas learned on this course will provide the conceptual foundation
for any further courses in quantitative methods that you may take.
The basic ideas will then not need to be learned from scratch
again, and the other courses can instead concentrate on introducing
further, ever more powerful statistical methods for different types of
data.
\end{itemize}

\section{Some basic definitions}
\label{s_intro_definitions}

Like any discipline, statistics involves some special terminology which
makes it easier to discuss its concepts with sufficient precision. Some
of these terms are defined in this section, while others will be
introduced later when they are needed.

You should bear in mind that all terminology is arbitrary, so there may
be different terms for the same concept. The same is true of notation
and symbols (such as $n$, $\mu$, $\bar{Y}$, $R^{2}$, and others) which
will be introduced later. Some statistical terms and symbols are so well
established that they are almost always used in the same way, but for
many others there are several versions in common use. While we try to be
consistent with the notation and terminology within this coursepack, we
cannot absolutely guarantee that we will not occasionally use different
terms for the same concept even here. In other textbooks
and in research articles you will certainly occasionally
encounter alternative terminology for some of these concepts. If you
find yourself confused by such differences, please come to the advisory
hours or ask your class teacher for clarification.

\subsection{Subjects and variables}
\label{ss_intro_def_subj}

Table \ref{t_datamatrix} shows a small set of quantitative data. Once
collected, the  data are typically arranged and stored in this kind of
spreadsheet-type rectangular table, known as a \textbf{data
matrix}. In the computer classes
you will see data in this form in
SPSS.

\begin{table}
\caption{An example of a small data matrix, showing measurements of
seven variables for 20 respondents in a social survey. The variables
are defined as \emph{age}: age in years;
\emph{sex}: sex (1=male; 2=female);
\emph{educ}: highest year of school completed;
\emph{wrkstat}: labour force status (1=working full time; 2=working part time;
3=temporarily not working; 4=unemployed; 5=retired; 6=in
education; 7=keeping house; 8=other);
\emph{life}: is life exciting or dull? (1=dull; 2=routine; 3=exciting);
\emph{income4}: total annual family income (1=\$24,999 or less;
2=\$25,000--\$39,999;
3=\$40,000--\$59,999;
4=\$60,000 or more;
99 indicates a missing value);
\emph{pres92}: vote in the 1992 presidential election
(0=did not vote or not eligible to vote; 1=Bill Clinton; 2=George H.\
W.\ Bush; 3=Ross Perot; 4=Other).
}
\label{t_datamatrix}
\begin{center}
\begin{tabular}{|r|r|r|r|r|r|r|r|}\hline
Id & \emph{age} & \emph{sex} & \emph{educ} & \emph{wrkstat} &
\emph{life} & \emph{income4} & \emph{pres92} \\
\hline \hline
1&43&1&11&1&2&3&2\\\hline
2&44&1&16&1&3&3&1\\\hline
3&43&2&16&1&3&3&2\\\hline
4&78&2&17&5&3&4&1\\\hline
5&83&1&11&5&2&1&1\\\hline
6&55&2&12&1&2&99&1\\\hline
7&75&1&12&5&2&1&0\\\hline
8&31&1&18&1&3&4&2\\\hline
9&54&2&18&2&3&1&1\\\hline
10&23&2&15&1&2&3&3\\\hline
11&63&2&4&5&1&1&1\\\hline
12&33&2&10&4&3&1&0\\\hline
13&39&2&8&7&3&1&0\\\hline
14&55&2&16&1&2&4&1\\\hline
15&36&2&14&3&2&4&1\\\hline
16&44&2&18&2&3&4&1\\\hline
17&45&2&16&1&2&4&1\\\hline
18&36&2&18&1&2&99&1\\\hline
19&29&1&16&1&3&3&1\\\hline
20&30&2&14&1&2&2&1\\\hline
\multicolumn{8}{l}{{\footnotesize Based on data from the U.S.\ General
Social Survey (GSS)}}
\end{tabular}
\end{center}
\end{table}

%\clearpage
The rows (moving downwards) and columns (moving left to right) of a data
matrix correspond to the first two important terms: the rows
to the \emph{subjects} and the columns to the \emph{variables} in the
data.
\begin{itemize}
\item
A \textbf{subject} is the smallest unit yielding information in the
study. In the example of Table \ref{t_datamatrix}, the subjects are
individual people, as they are in very many
social science examples. In other cases they may instead be families,
companies, neighbourhoods, countries, or whatever else is relevant in a
particular study. There is also much variation in the term itself, so
that instead of ``subjects'', a study might refer to ``units'',
``elements'', ``respondents'' or ``participants'', or simply to
``persons'', ``individuals'', ``families'' or ``countries'', for
example. Whatever the term, it is usually clear from the context what
the subjects are in a particular analysis.

The subjects in the data of Table \ref{t_datamatrix} are uniquely
identified only by a number (labelled ``Id'') assigned by the
researcher, as in a survey like this their names would not
typically be recorded. In situations where the identities of individual
subjects are available and of interest (such as when they are countries),
their names would typically be included in the data matrix.
\item
A \textbf{variable} is a characteristic which varies between subjects. For
example, Table \ref{t_datamatrix} contains data on seven variables ---
age, sex, education, labour force status, attitude to life,
family income and vote in a past election --- defined and recorded in the
particular ways explained in the caption of the table. It can be seen
that these are indeed ``variable'' in that not everyone has the same
value of any of them. It is this variation that makes collecting data on
many subjects necessary and worthwhile. In contrast, research questions
about characteristics which are the same for every subject (i.e.\
\emph{constants} rather than variables) are rare, usually not
particularly interesting, and not very difficult to answer.

The labels of the columns in Table \ref{t_datamatrix} (\emph{age},
\emph{wrkstat}, \emph{income4} etc.) are the names by which the
variables are uniquely identified in the data file on a computer. Such
concise titles are useful for this
purpose, but should be avoided when reporting the results of data
analyses, where clear English terms can be used instead. In other words,
a report should not say something like ``The analysis suggests that
WRKSTAT of the respondents is...'' but instead something like ``The
analysis suggests that the labour force status of the respondents
is...'', with the definition of this variable and its categories
also clearly stated.
\end{itemize}
\vspace*{-2ex}
Collecting quantitative data involves determining the values of a set of
variables for a group of subjects and assigning numbers to these values.
This is also known as \textbf{measuring} the values of the
variables. Here the word ``measure'' is used in a broader sense than in
everyday language, so that, for example, we are measuring a person's sex
in this sense when we assign a variable called ``Sex'' the value 1 if
the person is male and 2 if she is female. The value assigned to a
variable for a subject is called a \textbf{measurement} or an
\textbf{observation}. Our data thus consist of the measurements of a
set of variables for a set of subjects. In the data matrix, each row
contains the measurements of all the variables in the data for one
subject, and each column contains the measurements of one variable for
all of the subjects.

The number of subjects in a set of data is known as the \textbf{sample
size}, and is typically denoted by $n$. In a survey, for example,
this would be the number of people who responded to the questions in the
survey interview. In Table \ref{t_datamatrix} we have $n=20$. This would normally
be a very small sample size for a survey, and indeed the real
sample size in this one is several thousands. The twenty subjects here
were drawn from among them to obtain a small example which fits on a page.

A common problem in many studies is \textbf{nonresponse} or
\textbf{missing data}, which occurs when some measurements are not
obtained. For example, some survey respondents may refuse to answer
certain questions, so that the values of the variables corresponding to
those questions will be missing for them. In Table \ref{t_datamatrix},
the income variable is missing for subjects 6 and 18, and recorded only
as a \emph{missing value code}, here ``99''. Missing values create a
problem which has to be addressed somehow before or during the
statistical analysis. The easiest approach is to simply ignore all the
subjects with missing values and use only those with complete data on
all the variables needed for a given analysis.
For example, any analysis of the data in Table \ref{t_datamatrix} which
involved the variable \emph{income4} would then
exclude all the data for subjects 6 and 18. This method of
``complete-case analysis'' is usually applied
automatically by most statistical software packages, including SPSS. It
is, however, not a very good approach. For example, it means that a lot
of information will be thrown away if there are many subjects with some
observations missing. Statisticians have developed better ways of
dealing with missing data, but they are unfortunately beyond the scope
of this course.

\subsection{Types of variables}
\label{ss_intro_def_vartypes}

Information on a variable consists of the observations
(measurements) of it for the subjects in our data, recorded in the form
of numbers. However, not all numbers are the same. First, a particular
way of measuring a variable may or may not provide a good measure of the
concept of interest. For example, a measurement of a person's weight
from a well-calibrated scale would typically be a good measure
of the person's true weight, but an answer to the survey question ``How many
units of alcohol did you drink in the last seven days?'' might be a much
less accurate measurement of the person's true alcohol consumption (i.e.\
it might have \emph{measurement error} for a variety of reasons). So
just because you have put a number on a concept does not automatically
mean that you have captured that concept in a useful way. Devising good
ways of measuring variables is a major part of research design.
For example, social scientists are often interested in studying
attitudes, beliefs or personality traits, which are very difficult to
measure directly. A common approach is to develop \emph{attitude
scales}, which combine answers to multiple questions (``items'') on the
attitude into one number.

Here we will again leave questions of measurement to courses on research design, effectively assuming that the
variables we are analysing have been measured well enough for the
analysis to be meaningful. Even then we will have to consider some
distinctions between different kinds of variables. This is because the
type of a variable largely determines which methods of statistical analysis are
appropriate for that variable. It will be necessary to consider two related
distinctions:
\begin{itemize}
\item
Between different measurement levels
\item
Between continuous and  discrete variables
\end{itemize}

%\vspace*{-4ex}
\subsubsection{Measurement levels}
\label{sss_intro_def_vars_measlevels}

When a numerical value of a particular variable is
allocated to a subject, it becomes possible to relate that value to
the values assigned to other subjects. The \textbf{measurement level} of the
variable indicates how much information the number provides for such
comparisons. To introduce this concept, consider the variables obtained
as answers to the following three questions in the former U.K.\ General
Household Survey:\label{p_varexample}

[1] \emph{Are you}\\
\begin{tabular}{ll}
\emph{single, that is, never married?} & (coded as 1)\\
\emph{married and living with your husband/wife?} & (2)\\
\emph{married and separated from your husband/wife?} & (3)\\
\emph{divorced?} & (4)\\
\emph{or widowed?} & (5)
\end{tabular}

[2] \emph{Over the last twelve months, would you say your health has on
the whole been good, fairly good, or not good?}\\
(``Good'' is coded as 1, ``Fairly Good'' as 2, and ``Not Good'' as 3.)

[3] \emph{About how many cigaretters A DAY do you usually smoke on
weekdays?}\\
(Recorded as the number of cigarettes)

These variables illustrate three of the four possibilities
in the most common classification of measurement levels:
\begin{itemize}
\item
A variable is measured on a \textbf{nominal scale} if the numbers are
simply labels for different possible values (\emph{levels} or
\emph{categories}) of the variable. The only possible comparison is then
to identify whether two subjects have the \emph{same} or
\emph{different} values of the variable. The marital status variable [1]
is measured on a nominal scale. The values of such \emph{nominal-level
variables} are not in any order, so we cannot talk about one subject
having ``more'' or ``less'' of the variable than another subject; even
though ``divorced'' is coded with a larger number (4) than ``single''
(1), divorced is not more or bigger than single in any relevant sense.
We also cannot carry out arithmetical calculations on the values, as if
they were numbers in the ordinary sense. For example, if one person is
single and another widowed, it is obviously nonsensical to say that they
are on average separated (even though $(1+5)/2=3$).

The only requirement for the codes assigned to the levels of a
nominal-level variable is that different levels must receive different
codes. Apart from that, the codes are arbitrary, so that we can use any set of
numbers for them in any order. Indeed, the codes do not even need to be
numbers, so they may instead be displayed in the data matrix as short
words (``labels'' for the categories). Using successive small whole
numbers ($1,2,3,\dots$) is just a simple and concise choice for the
codes.

Further examples of nominal-level variables are the variables
\emph{sex}, \emph{wrkstat}, and \emph{pres92} in Table
\ref{t_datamatrix}.
\item
A variable is measured on an \textbf{ordinal scale} if its values do
have a natural ordering. It is then possible to determine not only
whether two subjects have the same value, but also whether one or the
other has a \emph{higher} value. For example, the self-reported health
variable [2] is an ordinal-level variable, as larger values indicate
worse states of health. The numbers assigned to the categories now have
to be in the correct order, because otherwise information about the true
ordering of the categories would be distorted. Apart from the order, the
choice of the actual numbers is still arbitrary, and calculations on
them are still not strictly speaking meaningful.

Further examples of ordinal-level variables are
\emph{life} and \emph{income4} in Table \ref{t_datamatrix}.
\item
A variable is measured on an
\textbf{interval scale} if \emph{differences} in its values are
comparable. One example is temperature measured on the Celsius
(Centigrade) scale. It is now meaningful to state not only that
20$^{\circ}$C is a \emph{different} and \emph{higher} temperature than
5$^{\circ}$C, but also that the \emph{difference} between them is
15$^{\circ}$C, and that that difference is of the same size as the
difference between, say, 40$^{\circ}$C and 25$^{\circ}$C. Interval-level
measurements are ``proper'' numbers in that calculations such as the
average noon temperature in London over a year are meaningful. What we
\emph{cannot} do is to compare \emph{ratios} of interval-level
variables. Thus 20$^{\circ}$C is not four times as warm as 5$^{\circ}$C,
nor is their real ratio the same as that of 40$^{\circ}$C and
10$^{\circ}$C. This is because the zero value of the Celcius scale
(0$^{\circ}$C) is not the lowest possible temperature but
an arbitrary point chosen for convenience of definition.
\item
A variable is measured on a \textbf{ratio scale} if it has all the
properties of an interval-level variable and also a true zero point.
For example, the smoking variable [3] is measured on a ratio level,
with zero cigarettes as its point of origin.
It is now possible to carry out all the comparisons possible for
interval-level variables, and also to compare ratios. For
example, it is meaningful to say that someone who smokes 20 cigarettes
a day smokes \emph{twice} as many cigarettes as one who smokes 10
cigarettes, and that that ratio is equal to the ratio of 30 and 15
cigarettes.

Further examples of ratio-level variables are \emph{age} and \emph{educ}
in Table \ref{t_datamatrix}.
\end{itemize}
The distinction between interval-level and ratio-level variables is in
practice mostly unimportant, as the same statistical methods can be
applied to both. We will thus consider them together throughout this
course, and will, for simplicity, refer to variables on either scale
as interval level
variables.
Doing so is logically coherent, because ratio level variables
have all the properties of interval level variables, as well the
additional property of a true zero point.

Similarly, nominal and ordinal variables can often be analysed with the
same methods. When this is the case, we will refer to them together as
nominal/ordinal level variables. There are, however, contexts where the
difference between them matters, and we will then discuss nominal and
ordinal scales separately.

The simplest kind of nominal variable is one with only \emph{two}
possible values, for example sex recorded as ``male'' or ``female'' or
an opinion recorded just as ``agree'' or ``disagree''. Such a variable
is said to be \textbf{binary} or \textbf{dichotomous}. As with any
nominal variable, codes for the two levels can be assigned in any way we
like (as long as different levels get different codes), for example as
1=Female and 2=Male; later it will turn out that in some analyses it is
most convenient to use the values 0 and 1.

The distinction between ordinal-level and interval-level variables
is sometimes further blurred in practice. Consider, for example, an
attitude scale of the kind mentioned above, let's say a scale for
happiness. Suppose that the possible values of the scale range from 0
(least happy) to 48 (most happy). In most cases it would be most
realistic to consider these measurements to be on an ordinal rather than
an interval scale. However, statistical methods developed specifically
for ordinal-level variables do not cope very well with variables with
this many possible values. Thus ordinal variables with many possible
values (at least more than ten, say) are typically treated as if they
were measured on an interval scale.


\subsubsection{Continuous and discrete variables}
\label{sss_intro_def_vars_cont}

This distinction is based on
the possible values a variable can have:
\begin{itemize}
\item
A variable is \textbf{discrete} if its basic unit of measurement cannot
be subdivided. Thus a discrete variable can only have certain values,
and the values between these are logically impossible.
For example, the marital status variable [1] and the health
variable [2] defined on page \pageref{p_varexample} are discrete,
because values like marital status of 2.3 or self-reported health of 1.7
are impossible given the way the variables are defined.
\item
A variable is \textbf{continuous} if it can in principle take infinitely
varied fractional values. The idea implies an unbroken scale or
continuum of possible
values. Age is an example of a continuous variable, as we can in
principle measure it to any degree of accuracy we like --- years, days,
minutes, seconds, micro-seconds. Similarly, distance, weight and even
income can be considered to be continuous.
\end{itemize}
You should note the ``in principle'' in this definition of continuous
variables above. Continuity is here a pragmatic
concept, not a philosophical one. Thus we will treat age and
income as continous even though they are in practice measured to the
nearest year or the nearest hundred pounds, and not in microseconds or
millionths of a penny (nor is the definition inviting you to start
musing on quantum mechanics and arguing that nothing is fundamentally
continuous). What the distinction between discrete and continuous really
amounts to in practice is the difference between variables which in our
data tend to take relatively few values (discrete variables) and ones
which can take lots of different values (continuous variables). This
also implies that we will sometimes treat variables which are undeniably
discrete in the strict sense as if they were really continuous. For
example, the number of people is clearly discrete when it refers to
numbers of registered voters in households (with a limited
number of possible values in practice), but effectively continuous when
it refers to populations of countries (with very many
possible values).

The measurement level of a variable refers to the way a characteristic
is recorded in the data, not to some other, perhaps more fundamental
version of that characteristic. For example, annual income
recorded to the nearest dollar is continuous, but an income variable
(c.f.\ Table \ref{t_datamatrix}) with values
\begin{itemize}
\item[1]
if annual income is \$24,999 or less;
\item[2]
if annual income is \$25,000--\$39,999;
\item[3]
if annual income is
\$40,000--\$59,999;
\item[4]
if annual income is
\$60,000 or more
\end{itemize}
\label{p_incomegr}
is discrete. This kind of variable, obtained by grouping ranges of
values of an initially continuous measurement, is common in the
social sciences, where the exact values of such variables are often not
that interesting and may not be very accurately measured.

The term \textbf{categorical variable} will be used in this coursepack
to refer to a discrete variable which has only a finite (in practice
quite small) number of possible values, which are known in advance. For
example, a person's sex is typically coded simply as
``Male'' or ``Female'', with no other values. Similarly, the
grouped income variable shown above is categorical, as every income
corresponds to one of its four categories (note that it is the ``rest''
category 4 which guarantees that the variable does indeed cover all
possibilities). Categorical variables are of separate interest because
they are common and because some statistical methods are designed
specifically for them.
An example of a non-categorical discrete
variable is the population of a country, which does not have a small,
fixed set of possible values (unless it is again transformed into
a grouped variable as in the income example above).

\subsubsection{Relationships between the two distinctions}
\label{sss_intro_def_vars_rels}

The distinctions between variables with different measurement levels on
one hand, and continuous and discrete variables on the other, are
partially related. Essentially all nominal/ordinal-level variables are
discrete, and almost all continous variables are interval-level variables. This
leaves one further possibility, namely a discrete interval-level
variable; the most common example of this is a \textbf{count}, such as
the number of children in a family or the population of a country.
These connections are summarized in Table \ref{t_vartypes}.

\begin{table}
\caption{Relationships between the types of variables discussed in
Section
\ref{ss_intro_def_vartypes}.}
\label{t_vartypes}
\vspace*{2ex}
\begin{tabular}{l|l|l|}
\multicolumn{1}{l}{ }
& \multicolumn{2}{c}{\underline{Measurement level}} \\
\multicolumn{1}{l}{ }
& \multicolumn{1}{l}{\rule[-3mm]{0mm}{8mm}\textbf{Nominal/ordinal}} &
\multicolumn{1}{l}{\textbf{Interval/ratio}} \\ \cline{2-3}
\textbf{Discrete} &
Many & \emph{Counts} \\
$\bullet$ Always \textbf{categorical}, &
$\bullet$ If many different \\
i.e.\ having a fixed set &
observed values, \\
&
of possible values &
often treated as \\
&
(categories) &
effectively continuous \\
&
$\bullet$ If only two categories, & \\
&
variable is \textbf{binary} & \\
&
(\textbf{dichotomous}) & \\[1ex] \cline{2-3}
\textbf{Continuous} & None & Many \\
\cline{2-3}
\end{tabular}
%\vspace*{1ex}
\end{table}


In practice the situation may be even simpler than this, in that
the most relevant distinction is often between the following two
cases:\label{p_2_variable_types}
\begin{enumerate}
\item
Discrete variables with a small number of observed values. This includes
both categorical variables, for which all possible values are known in
advance, and variables for which only a small number of values were
actually observed even if others might have been possible\footnote{
For example, suppose we collected data on the number of traffic
accidents on each of a sample of streets in a week, and suppose
that the only numbers observed were 0, 1, 2, and 3. Other,
even much larger values were clearly at least logically possible, but
they just did not occur. Of course, redefining the largest value as ``3
or more'' would turn the variable into an unambiguously
categorical one.}.
Such variables can be conveniently summarized
in the form of tables and handled by methods appropriate for such
tables, as described later in this coursepack.
This group also includes all nominal variables,
even ones with a relatively large number of categories, since methods
for group 2.\ below are entirely inappropriate for them.
\item
Variables with a large number of possible values. This includes all
continuous variables and those interval-level or ordinal discrete
variables which have so many values that it is pragmatic to treat them as effectively
continuous.
\end{enumerate}
Although there are contexts where we need to distinguish between types
of variables more carefully than this, for practical purposes
this simple distinction is often sufficient.

\subsection{Description and inference}
\label{ss_intro_def_descr}

In the past, the subtitle of this course was ``Description and
inference''. This is still descriptive
of the contents of the course. These words
refer to two different although related tasks of statistical
analysis. They can be thought of as solutions to what might be called
the ``too much and not enough'' problems with observed data. A set of
data is ``too much'' in that it is very difficult to understand or
explain the data,  or to draw any conclusions from it, simply by staring
at the numbers in a data matrix. Making much sense of even a small data
matrix like the one in Table \ref{t_datamatrix} is challenging, and the
task becomes entirely impossible with bigger ones.
There is thus a clear need for methods of statistical
description:
\begin{itemize}
\item
\textbf{Description}: summarizing some features of the data in ways that
make them easily understandable. Such methods of description may be in
the form of numbers or graphs.
\end{itemize}
The ``not enough'' problem is that quite often the subjects in
the data are treated as representatives of some larger group which is
our real object of interest. In statistical terminology, the observed
subjects are regarded as a \textbf{sample} from a larger
\textbf{population}.
For example, a pre-election opinion poll is not carried out because we are
particularly interested in the voting intentions of the particular
thousand or so people who answer the questions in the poll (the sample),
but because we hope that their answers will help us draw
conclusions about the preferences of all of those who intend to
vote on election day (the population). The job of statistical
inference is to provide methods for generalising from a sample to the
population:
\begin{itemize}
\item
\textbf{Inference}: drawing conclusions about characteristics of a
population based on the data observed in a sample. The two main
tools of statistical inference are \textbf{significance tests} and
\textbf{confidence intervals}.
\end{itemize}
Some of the methods described on this course are mainly intended for
description and others for inference, but many also have a useful
role in both.

\subsection{Association and causation}
\label{ss_intro_def_assoc}

The simplest methods of analysis described on this course consider
questions which involve only one variable at a time. For example, the
variable might be the political party a respondent intends to vote for
in the next general election. We might then want to know what proportion
of voters plan to vote for the Labour party, or which party is likely
to receive the most votes.

However,
considering variables one at a time is not going
to entertain us for very long. This is
because most interesting research questions involve associations between
variables. One way to define an association is
that
\begin{itemize}
\item
There is an \textbf{association} between two variables if knowing the
value of one of the variables will help to predict the value of the
other variable.
\end{itemize}
(A more careful definition will be given later.)
Other ways of referring to the same concept are that the variables are
``related'' or that there is a ``dependence'' between them.

For example, suppose that instead of considering voting intentions
overall, we were interested in \emph{comparing} them between two groups
of people, homeowners and people who live in rented accommodation.
Surveys typically suggest that homeowners are more likely to vote for
the Conservatives and less likely to vote for Labour than renters. There
is then an association between the two (discrete) variables ``type of
accommodation'' and ``voting intention'', and knowing the type of a
person's accommodation would help us better predict who they intend to
vote for. Similarly, a study of education and income might find that
people with more education (measured by years of education completed)
tend to have higher incomes (measured by annual income in pounds), again
suggesting an association between these two (continuous) variables.

Sometimes the variables in an association are in some sense on an equal
footing. More often, however, they are instead considered asymmetrically
in that it is more natural to think of one of them as being used to
predict the other. For example, in the examples of the previous
paragraph it seems easier to talk about home ownership predicting voting
intention than vice versa, and of level of education predicting income
than vice versa. The variable used for prediction is then known as an
\textbf{explanatory variable} and the variable to be predicted as the
\textbf{response variable} (an alternative convention is to talk about
\textbf{independent} rather than explanatory variables and
\textbf{dependent} instead of response variables). The most powerful
statistical techniques for analysing associations between explanatory
and response variables are known as \textbf{regression} methods. They
are by far the most important family of methods of quantitative data
analysis. On this course you will learn about the most important member
of this family, the method of \textbf{linear regression}.

In the many research questions where regression methods are useful, it
almost always turns out to be crucially important to be able to consider
several different explanatory variables simultaneously for a single
response variable. Regression methods allow for this
through the techniques of \textbf{multiple regression}.

The statistical concept of association is closely related to the
stronger concept of \textbf{causation}, which is at the heart of very
many research questions in the social sciences and elsewhere. The two
concepts are not the same. In particular, association is not
\emph{sufficient} evidence for causation, i.e.\ finding that two
variables are statistically associated does not prove that either
variable has a causal effect on the other. On the other hand,
association is almost always \emph{necessary} for causation: if
there is no association between two variables, it is very unlikely that
there is a direct causal effect between them. This means that analysis
of associations is a necessary part, but not the only part, of the
analysis of causal effects from quantitative data. Furthermore,
statistical analysis of associations is carried out in essentially the
same way whether or not it is intended as part of a causal argument. On
this course we will mostly focus on associations. The kinds of additional
arguments that are needed to support causal conclusions are based on
information on the
research design and the nature of the variables. They are discussed only
briefly on this course, and at greater length on courses of research design such as
MY400 (and the more advanced MY457, which considers design and analysis
for causal inference together).


\section{Outline of the course}
\label{s_intro_outline}

We have now defined three separate distinctions between different
problems for statistical analysis, according to (1) the types of
variables involved, (2) whether description or inference is required,
and (3) whether we are examining one variable only or associations
between several variables. Different combinations of these elements
require different methods of statistical analysis. They also provide the
structure for the course, as follows:
\begin{itemize}
\item
\textbf{Chapter \ref{c_descr1}}: Description for single variables of any
type, and for associations between categorical variables.
\item
\textbf{Chapter \ref{c_samples}}: Some general concepts
of statistical inference.
\item
\textbf{Chapter \ref{c_tables}}:
Inference for associations between categorical variables.
\item
\textbf{Chapter \ref{c_probs}}:
Inference for single dichotomous variables, and
for associations between a dichotomous explanatory
variable and a dichotomous response variable.
\item
\textbf{Chapter \ref{c_contd}}: More general concepts
of statistical inference.
\item
\textbf{Chapter \ref{c_means}}:
Description and inference for associations between a
dichotomous explanatory
variable and a continuous response variable, and
inference for single continuous variables.
\item
\textbf{Chapter \ref{c_regression}}:
Description and inference for associations between any kinds of
explanatory variables and a continuous response variable.
\item
\textbf{Chapter \ref{c_3waytables}}: Some additional comments on analyses which
involve three or more categorical variables.
\end{itemize}
As well as in Chapters \ref{c_samples} and \ref{c_contd}, general
concepts of statistical inference are also gradually introduced in
Chapters \ref{c_tables}, \ref{c_probs} and \ref{c_means}, initially in
the context of the specific analyses considered in these chapters.

\section{The use of mathematics and computing}
\label{s_intro_maths}

Many of you will approach this course with some reluctance and
uncertainty, even anxiety. Often this is because of fears about
mathematics, which may be something you never liked or never learned
that well. Statistics does indeed involve a lot of mathematics in both
its algebraic (symbolical) and arithmetic (numerical) senses. However,
the understanding and use of statistical concepts and methods can be
usefully taught and learned even without most of that mathematics, and
that is what we hope to do on this course. It is perfectly possible to
do well on the course without being at all good at mathematics of the
secondary school kind.

\subsection{Symbolic mathematics and mathematical notation}

Statistics \emph{is} a mathematical subject in that its concepts and
methods are expressed using mathematical formalism, and
grounded in a branch of mathematics known as probability theory. As a
result, heavy use of mathematics is essential for those who develop
these methods (i.e.\ statisticians). However, those who only \emph{use}
them (i.e.\ you) can ignore most of it and still gain a solid and
non-trivialised understanding of the methods. We will thus be able to
omit most of the mathematical details. In particular, we will not show
you how the methods are derived or prove theorems about them, nor do we
expect you to do anything like that.

We will, however, use mathematical notation whenever necessary to state
the main results and to define the methods used. This is because
mathematics is the language in which many of these results are easiest
to express clearly and accurately, and trying to avoid all mathematical
notation would be contrived and unhelpful. Most of the notation is
fairly simple and will be explained in detail. We will also interpret
such formulas in English as well to draw attention to their most
important features.

Another way of explaining statistical methods is through applied
examples. These will be used throughout the course. Most of them are
drawn from real data from research in a range social of social sciences.
If you wish to find further examples of how these methods are used in
your own discipline, a good place to start is in relevant books and
research journals.

\subsection{Computing}

Statistical analysis involves also a lot of mathematics of the numerical
kind, i.e.\ various calculations on the numbers in the data. Doing such
calculations by hand or with a pocket calculator would be tedious and
unenlightening, and in any case impossible for all but the smallest
samples and simplest methods. We will mostly avoid doing that by leaving
the drudgery of calculation to computers, where the methods are
implemented in statistical software packages. This also means that you
can carry out the analyses without understanding all the numerical
details of the calculations. Instead, we can focus on trying to
understand when and why certain methods of analysis are used, and
learning to interpret their results.

A simple pocket calculator is still more convenient than a
computer for some very simple calculations. You will also need one for
this purpose in the examination, where computers are not allowed. Any
such calculations required in the examination will be extremely simple
to do (assuming you know what you are trying to do, of course). For more
complex analyses, the exam questions will involve interpreting computer
output rather than carrying out the calculations. The homework questions
that follow the computer classes contain examples of both of these types of
questions.

The software package used in the computer classes of this course is
called SPSS. There are other comparable packages, for
example SAS, Minitab, Stata and R. Any one of them could be used for the
analyses on this course, and the exact choice does not matter very much.
SPSS is convenient for our purposes, because it is widely used,
has a reasonably user-friendly menu interface, and is available on a
cheap licence even for the personal computers of LSE students.

Sometimes you may see a phrase such as ``SPSS course'' used apparently
as a synonym for ``Statistics course''. This makes as little sense as
treating an introduction to Microsoft Word as a course on how to write
good English. It is not possible to learn quantitative data analysis
well by just sitting down in front of SPSS or any other statistics
package and trying to figure out what all those menus are for. On the
other hand, using SPSS to apply statistical methods to analyse real data
is an effective way of strengthening the understanding of
those methods \emph{after} they have first been introduced in lectures.
That is why this course has weekly computer classes.

The software-specific questions on how to carry out statistical analyses are
typically of a lesser order of difficulty once the methods
themselves are reasonably well understood. In other words, once you have
a clear idea of what you want to do, finding out how to do it in SPSS
tends not to be that difficult. For example, in the next chapter we will
discuss the mean, one simple tool of descriptive statistics. Suppose
that you then want to calculate the mean of a variable called \emph{Age}
in a data set. Learning how to do this in SPSS is then a matter of (1)
finding the menu item where SPSS can be told to calculate a mean, (2)
finding out which part of that menu is used to tell SPSS that you want
the mean of \emph{Age} specifically, and (3) finding the part of the
SPSS output where the calculated mean of \emph{Age} is reported.
Instructions for steps like this for techniques covered on this course
are given in the descriptions of the corresponding computer classes.

There are, however, some tasks which have more to do with specific
software packages than with statistics in general. For example, the fact
that SPSS has a menu interface, and the general style of those menus,
need to be understood first. You also need to learn how to get data into
SPSS in the first place, how to manipulate the data in various ways, and
how to export output from the analyses to other packages. Some
instructions on how to do such things are given in the first computer
class. The introduction to the computer classes on page
\pageref{c_class0} also includes details of some SPSS guidebooks and
other sources of information which you may find useful if you want to
know more about the program.
