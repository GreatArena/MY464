\chapter{More statistics...}
\label{c_more}

You will no doubt be pleased to learn that the topics covered on this
course have not quite exhausted the list of available statistical
methods. In this chapter we outline some of the most important
further areas of statistics, so that you are at least aware of their
existence and titles. For some of them, codes of LSE courses which cover
these methods are given in parentheses.

A very large part of advanced statistics is devoted to further types of
\textbf{regression models}. The basic idea of them is the same as for
multiple linear regression, i.e.\ modelling expected values of response
variables given several explanatory variables. The issues involved in
the form and treatment of explanatory variables are usually almost
exactly the same as for linear models. Different classes of regression
models are needed mainly to accommodate different types of response
variables:
\begin{itemize}
\item
Models for \textbf{categorical response variables}. These exist
for situations where the response
variable is dichotomous (\textbf{binary regression}, especially
\textbf{logistic models}), has more than two unordered
(\textbf{multinomial logistic models})
or ordered (\textbf{ordinal regression
models}) categories, or is a count, for example in a contingency table
(\textbf{Poisson regression}, \textbf{loglinear models}).
Despite the many different titles, all of
these models are closely connected
(MY452)
\item
Models for cases where the response is a length of time to some event,
such as a spell of unemployment, interval between births of children
or survival of
a patient in a medical study. These techniques are known as
\textbf{event history analysis}, \textbf{survival analysis} or
\textbf{lifetime data analysis}. Despite the different terms,
all refer to the same statistical models.
\end{itemize}

Techniques for the analysis of \textbf{dependent data}, which do not
require the assumption of statistically independent observations used by
almost all the methods on this course:
\begin{itemize}
\item
\textbf{Time series analysis} for one or more long sequence of
observations of the same quantity over time. For example, each of the
five temperature sequencies in Figure \ref{f_temperatures} is a time
series of this kind.
\item
Regression models for \textbf{hierarchical data}, where some sets of
observations are not independent of each other. There are two main types
of such data: \textbf{longitudinal} or \textbf{panel data} which consist
of short time series for many units (e.g.\ answers
by respondents in successive waves of a panel survey), and \textbf{nested} or
\textbf{multilevel data} where basic units are grouped in natural
groups or clusters (e.g.\ pupils in classes and schools in an
educational study). Both of these can be analysed using the same general
classes of models, which in turn are generalisations of linear and other
regression models used for independent data
(ST416 for models for multilevel data and ST442 for models for
longitudinal data).
\end{itemize}

Methods for \textbf{multivariate data}. Roughly speaking,
this means data with several variables for comparable
quantities treated on an equal footing, so that
none of them is obviously a response to the others. For example, results
for the ten events in the decathlon data of the week 7 computer class or, more seriously,
the responses to a series of related attitude items in a survey are
multivariate data of this kind.
\begin{itemize}
\item
Various methods of \textbf{descriptive multivariate analysis} for
jointly summarising and presenting information on the many variables,
e.g.\
\textbf{cluster analysis}, \textbf{multidimensional scaling} and
\textbf{principal component analysis} (MY455 for principal components
analysis).
\item
Model-based methods for multivariate data. These are typically
\textbf{latent variable models}, which also
involve variables which can never be directly observed. The simplest
latent variable technique is
\textbf{exploratory factor analysis},
and others include \textbf{confirmatory factor analysis},
\textbf{structural equation models}, and \textbf{latent trait} and
\textbf{latent class} models (MY455).
\end{itemize}

Some types of \textbf{research design} may also involve particular
statistical considerations:
\begin{itemize}
\item
\textbf{Sampling theory} for the design of probability
samples, e.g.\  for surveys (part of MY456, which also covers
methodology of surveys in general).
\item
\textbf{Design of experiments} for more complex randomized experiments.
\end{itemize}

Finally, some areas of statistics are concerned with broader and more
fundamental aspects of statistical analysis, such as alternative forms
of model specification and inference (e.g.\ \textbf{nonparametric
methods}) or the basic ideas of inference itself (e.g.\ \textbf{Bayesian
statistics}). These and the more specific tools further build on the
foundations of all statistical methods, which are the subject of
\textbf{probability theory} and \textbf{mathematical statistics}.
However, you are welcome, if you wish, to leave the details of these
fields to professional statisticians, if only to keep them too in
employment.
