\chapter{Samples and populations}
\label{c_samples}

\section{Introduction}
\label{s_samples_intro}

So far we have discussed statistical description, which is concerned
with summarizing features of a sample of observed data. From now on,
most of the attention will be on statistical inference. As noted in
Section \ref{ss_intro_def_descr}, the purpose of inference is to draw
conclusions about the characteristics of some larger population based on
what is observed in a sample. In this chapter we will first give more
careful definitions of the concepts of populations and samples, and of
the connections between them. In Section \ref{s_samples_popdistrs} we
then consider the idea of a population distribution, which is the target
of statistical inference. The discussion of statistical inference will
continue in Chapters \ref{c_tables}--\ref{c_means} where we gradually
introduce the basic elements of inference in the contexts of different
types of analyses.

\section{Finite populations}
\label{s_samples_finpops}

In many cases the population of interest is a particular group of real people
or other units. Consider, for example, the European Social Survey (ESS)
which we used in Chapter \ref{c_descr1} (see page
\pageref{p_ess_example})\footnote{ European Social Survey (2012). ESS
Round 5 (2010/2011) Technical Report. London: Centre for Comparative
Social Surveys, City University London. See
\texttt{www.europeansocialsurvey.org} for more on the ESS.}. The ESS is
a cross-national survey carried out biennially in around 30 European
countries. It is an academically-driven social survey which is designed
to measure a wide range attitudes, beliefs and behaviour patterns among
the European population, especially for purposes for cross-national
comparisons.

The target population of ESS is explicitly stated as being ``all persons
aged 15 and over resident within private households, regardless of their
nationality, citizenship, language or legal status'' in each of the
participating countries. This is, once ``private household'' has been
defined carefully, and notwithstanding the inevitable ambiguity in that
the precise number and composition of households are constantly changing,
a well-defined, existing group. It is also a large group: in the
UK, for example, there are around 50~million such people. Nevertheless,
we have no conceptual difficulty with imagining this collection of
individuals. We will call any such population a \emph{finite
population}.

The main problem with studying a large finite population is that it
is usually not feasible to collect data on all of its members.
A \textbf{census} is a study where some variables \emph{are}
in fact measured for the entire population. The best-known example is
the Census of Population, which at least aims to be a complete
evaluation of all persons living in a country on a particular date with
respect to basic demographic data. Similarly, we have the Census of
Production, Census of Distribution etc. For most research, however, a
census is not feasible. Even when one is attempted, it is rarely truly
comprehensive. For example, all population censuses which involve
collecting the data from the people themselves end up missing a
substantial (and non-random) proportion of the population. For most purposes
a well-executed sample of the kind described below is actually
preferable to an unsuccessful census.

\section{Samples from finite populations}
\label{s_samples_samples}

When a census is not possible, information on the population is obtained
by observing only a subset of units from it, i.e.\ a
sample. This is meant to be \emph{representative} of the
population, so that we can \emph{generalise} findings from the sample to
the population. To be representative in a sense appropriate for
statistical inference, a sample from a finite population must be a \emph{probability sample},
obtained using
\begin{itemize}
\item
\textbf{probability sampling}: a sampling method where every unit in the
population has a \textbf{known}, \textbf{non-zero} probability of being
selected to the sample.
\end{itemize}
Probability sampling requires first a \textbf{sampling frame},
essentially one or more lists of units or collections of units which
make it possible to select and contact members of the sample. For
example, the first stage of sampling for many UK surveys uses the
Postcode Address File, a list of postal addresses in the country. A
\textbf{sampling design} is then created in such a way that it assigns a
\textbf{sampling probability} for each unit, and the sample is drawn so
that each unit's probability of being selected into the sample is given
by their sampling probability. The selection of the specific set of
units actually included in the sample thus involves \emph{randomness},
usually implemented with the help of random number generators on
computers.

The simplest form of probability sampling is
\begin{itemize}
\item
\textbf{simple random sampling}, where every unit in the population has
the \emph{same} probability of selection.
\end{itemize}
This requirement of equal selection probabilities is by no means
essential. Other probability sampling methods which relax it include
\begin{itemize}
\item
\textbf{stratified sampling}, where the selection probabilities are set
separately for different groups (\emph{strata})
in the population, for example separately for men and women, different
ethnic groups or people living in different regions.
\item
\textbf{cluster sampling}, where the units of interest
are not sampled individually but in groups (\emph{clusters}).
For example, a school survey
might involve sampling entire classes and then interviewing every pupil
in each selected class.
\item
\textbf{multistage sampling}, which employs a sequence of steps,
often with a combination of stratification, clustering and simple
random sampling. For example, many social surveys use a \emph{multistage
area sampling} design which begins with one or more stages of sampling
areas, then households (addresses) within selected small areas, and
finally individuals within selected households.
\end{itemize}
These more complex sampling methods are in fact used for most
large-scale social surveys to improve their accuracy and/or
cost-efficiency compared to simple random sampling. For example, the UK
component of the European Social Survey uses a design of three stages:
(1) a stratified sample of postcode sectors, stratified by region, level
of deprivation, percentage of privately rented households, and
percentage of pensioners; (2) simple random sample of addresses within
the selected sectors; and (3) simple random sample of one adult from
each selected address.

Some analyses of such data require the use of \emph{survey weights} to
adjust for the fact that some units were more likely than
others to end up in the sample. The questions of how and when the
weights should be used are, however, beyond the scope of this course.
Here we will omit the weights even in examples where they might normally
be used.\footnote{For more on survey weights and the design and analysis
of surveys in general, please see MY456 (Survey Methodology) in the Lent
Term.}

Not all sampling methods satisfy the requirements of probability
sampling. Such techniques of \textbf{non-probability sampling} include
\begin{itemize}
\item
\emph{purposive sampling}, where the investigator uses his or her own
``expert'' judgement to select units considered to be representative of
the population. It is very difficult to do this well, and very easy to
introduce conscious or unconscious biases into the selection. In general,
it is better to leave the task to the random processes of probability
sampling.
\item
\emph{haphazard} or \emph{convenience} sampling, as
when a researcher simply uses the first $n$ passers-by who happen to be
available and willing to answer questions. One version of this is
\emph{volunteer} sampling, familiar from call-in ``polls'' carried out
by morning television shows and newspapers on various topics of current
interest. All we learn from such exercises are the opinions of those
readers or viewers who felt strongly enough about the issue to send in
their response, but these tell us essentially nothing about the
average attitudes of the general population.
\item
\emph{quota sampling}, where interviewers are required to select a
certain number (quota) of respondents in each of a set of categories
(defined, for example, by sex, age group and income group). The
selection of specific respondents within each group is left to the
interviewer, and is usually done using some (unstated) form of purposive or
convenience sampling. Quota sampling is quite common, especially in
market research, and can sometimes give reasonable results. However, it
is easy to introduce biases in the selection stage, and almost
impossible to know whether the resulting sample is a representative one.
\end{itemize}

A famous example of the dangers of non-probability sampling is the
survey by the \emph{Literary Digest} magazine to predict the results of
the 1936 U.S.\ presidential election. The magazine sent out about 10
million questionnaires on post cards to potential respondents, and based
its conclusions on those that were returned. This introduced biases in
at least two ways. First, the list of those who were sent the
questionnaire was based on registers such as the subscribers to the
magazine, and of people with telephones, cars and various club
memberships. In 1936 these were mainly wealthier people who were more
likely to be Republican voters, and the typically poorer people not on
the source lists had no chance of being included. Second, only about
25\% of the questionnaires were actually returned, effectively rendering
the sample into a volunteer sample. The magazine predicted that the
Republican candidate Alf Landon would receive 57\% of the vote, when in
fact his Democratic opponent F.\ D.\ Roosevelt gained an overwhelming
victory with 62\% of the vote. The outcome of the election was predicted
correctly by a much smaller probability sample collected by George
Gallup.

A more recent example is the ``GM Nation'' public
consultation exercise on attitudes to genetically modified (GM)
agricultural products, carried out in the U.K. in 2002--3\footnote{For
more information, see Gaskell, G. (2004). ``Science policy and society:
the British debate over GM agriculture'', \emph{Current Opinion in
Biotechnology} \textbf{15}, 241--245.}. This involved various
activities, including national, regional and local events where
interested members of the public were invited to take part in
discussions on GM foods. At all such events the participants also
completed a questionnaire, which was also available on the GM Nation
website. In all, around 37000 people completed the questionnaire, and
around 90\% of those expressed opposition to GM foods. While the authors
of the final report of the consultation drew some attention to the
unrepresentative nature of this sample, this fact had certainly been
lost by the time the results were reported in the national newspapers as
``5 to 1 against GM crops in biggest ever public
survey''. At the same time, probability samples suggested that the
British public is actually about evenly split between supporters and
opponents of GM foods.

\section{Conceptual and infinite populations}
\label{s_samples_infpops}

Even a cursory inspection of academic journals in the social sciences
will reveal that a finite population of the kind discussed above is not
always clearly defined, nor is there often any reference to probability
sampling. Instead, the study designs may for example resemble the
following two examples:

\underline{\emph{Example: A psychological experiment}}\\
Fifty-nine undegraduate students from a large U.S.\ university took part
in a psychological experiment\footnote{Experiment 1 in Anderson, C.\ A.,
Carnagey, N.\ L., and Eubanks, J.\ (2003). ``Exposure to violent media:
the effects of songs with violent lyrics on aggressive thoughts and
feelings''. \emph{Journal of Personality and Social Psychology}
\textbf{84}, 960--971.}, either as part of a class project or for extra
credit on a psychology course. The participants were randomly assigned
to listen to one of two songs, one with clearly violent lyrics and one
with no violent content. One of the variables of interest was a measure
(from a 35-item attitude scale) of state hostility (i.e.\ temporary
hostile feelings), obtained after the participants had listened to a
song, and the researchers were interested in comparing levels of
hostility between the two groups.

\underline{\emph{Example: Voting in a congressional election}}\\
A political-science article\footnote{ Carson, J.\ L.\ et al.\ (2001).
``The impact of national tides and district-level effects on electoral
outcomes: the U.S.\ congressional elections of 1862--63''.
\emph{American J.\ of Political Science} \textbf{45}, 887--898.}
considered the U.S.\ congressional election which took place between
June 1862 and November 1863, i.e.\ during a crucial period in the
American Civil War. The units of analysis were the districts
in the House of Representatives. One part of the analysis examined whether the
likelihood of the candidate of the Republican Party (the party of the
sitting president Abraham Lincoln) being elected from a district was
associated with such explanatory variables as whether the Republican was
the incumbent, a measure of the quality of the other main candidate,
number of military casualties for the district, and the timing of the
election in the district (especially in relation to the Union armies'
changing fortunes over the period).

There is no reference here to the kinds of finite populations and
probability samples discussed Sections \ref{s_samples_finpops} and
\ref{s_samples_samples}. In the  experiment, the
participants were a convenience sample of respondents easily
available to the researcher, while in the election study the units
represent (nearly) all the districts in a single (and historically
unique) election. Yet both articles contain plenty of statistical
inference, so the language and concepts of samples and populations are
clearly being used. How is this to be justified?

In the example of the psychological experiment the subjects will clearly
not be representative of a general (non-student) population in many
respects, e.g.\ in age and education level. However, it
is not really such characteristics that the study is concerned with, nor
is the population of interest really a population of people. Instead,
the implicit ``population'' being considered is that of possible values
of level of hostility after a person has
listened to one of the songs in the experiment. In this extended
framework, these possible values include not just the levels of
hostitility possibly obtained for different people, but also those that
a single person might have after listening to the song at different
times or in different moods etc. The generalisation from the observed
data in the experiment is to this hypothetical population of possible
reactions.

In the political science example the population is also a hypothetical
one, namely those election results that \emph{could} have been obtained
if something had happened differently, i.e.\ if different people turned
up to vote, if some voters had made different decisions, and so on (or
if we considered a different election in the same conditions, although
that is less realistic in this example, since other elections have not
taken place in the middle of a civil war). In other words, votes that
actually took place are treated as a sample from the population of votes
that could conceivably have taken place.

In both cases the ``population'' is in some sense a hypothetical or
conceptual one, a population of possible realisations of events, and the
data actually observed are a sample from that population. Sometimes it
is useful to apply similar thinking even to samples from ostensibly
quite finite populations. Any such population, say the residents of a
country, is exactly fixed at one moment only, and was and will be
slightly different at any other time, or would be even now if any one of
a myriad of small events had happened slightly differently in the past.
We could thus view the finite population itself at a single moment as a
sample from a conceptual population of possible realisations. This is
known in survey literature as a \emph{superpopulation}. The data
actually observed are then also a sample from the superpopulation. With
this extension, it is possible to regard almost any set of data as a
sample from some conceptual superpopulation.

The highly hypothetical notion of a conceptual population of possible
events is clearly going to be less easy both to justify and to
understand than the concept of a large but finite population of real
subjects defined in Section \ref{s_samples_finpops}. If you find the
whole idea distracting, you can focus in your mind on the more
understandable latter case, at least if you are willing to believe that
the idea of a conceptual population is also meaningful. Its main
justification is that much of the time it works, in the sense that
useful decision rules and methods of analysis are obtained based on the
idea. Most of the motivation and ideas of statistical inference are
essentially the same for both kinds of populations.

Even when the idea of a conceptual population is invoked,
questions of representativeness of and generalisability to
real, finite populations will still need to be kept in mind in most
applications. For example, the assumption behind the psychological
experiment described above is that the findings about how hearing a
violent song affects levels of hostility are generalisable to some
larger population, beyond the 59 participants in the experiment and
beyond the body of students in a particular university. This may well be
the case at least to some extent, but it is still open to questioning.
For this reason findings from studies like this only become really
convincing when they are \emph{replicated} in comparable experiments
among different kinds of participants.

Because the kinds of populations discussed in this section are
hypothetical, there is no sense of them having a particular fixed number
of members. Instead, they are considered to be \emph{infinite} in size.
This also implies (although it may not be obvious why) that we can
essentially always treat samples from such populations as if they were
obtained using simple random sampling.

\section{Population distributions}
\label{s_samples_popdistrs}

We will introduce the idea of a population distribution first for finite
populations, before extending it to infinite ones. The discussion in
this section focuses on categorical variables, because the concepts are
easiest to explain in that context; generalisations to continuous
variables are discussed in Chapter \ref{c_means}.

Suppose that we have drawn a sample of $n$ units from a finite
population and determined the values of some variables for them. The
units that are not in the sample also possess values of the variables,
even though these are not observed. We can thus easily imagine how any
of the methods which were in Chapter \ref{c_descr1} used to describe a
sample could also be applied in the same way to the whole population, if
only we knew all the values in it. In particular, we can, paralleling
the sample distribution of a variable, define the \textbf{population
distribution} as the set of values of the variable which appear in the
population, together with the frequencies of each value.

For illustration, consider again the example introduced on page
\pageref{p_ess_example}. The two variables there are a
person's sex and his or her attitude toward income redistribution. We
have observed them for a sample $n=2344$ people drawn from the
population of all UK residents aged 15 or over. The sample distributions
are summarised by Table \ref{t_attitude} on page \pageref{t_attitude}.

\begin{table}
\caption{Attitude towards income redistribution by sex, in a hypothetical
population of 50 million people. The numbers in the table are
frequencies in millions of people, row percentages (in parentheses) and
overall percentages [in square brackets].}
\label{t_sex_attitude_pop}
\begin{center}
\begin{tabular}{|l|ccccc|r|}\hline
& \multicolumn{5}{|c|}{\emph{``The government should
take measures}} & \\
& \multicolumn{5}{|c|}{\emph{to reduce differences in income levels''}}
& \\[.3ex]
 & Agree & & Neither agree & & Disagree & \\
Sex & strongly & Agree & nor disagree & Disagree & strongly & Total \\ \hline
Male & 3.84 & 10.08 & 4.56 & 4.32 & 1.20  &  24.00\\
& (16.00) & (42.00) &(19.00) & (18.00) & (5.00) & (100) \\
& [7.68] & [20.16] & [9.12] & [8.64] & [2.40] & [48.00] \\[.5ex]
Female & 4.16 & 13.00 & 4.68 & 3.38 & 0.78 & 26.00\\
& (16.00) & (50.00) & (18.00) & (13.00) & (3.00) & (100)\\
& [8.32] & [26.00] & [9.36] & [6.76] & [1.56] & [52.00] \\
\hline
Total & 8.00 & 23.08 &  9.24&  7.70& 1.98 & 50 \\
      & (16.00) & (46.16) & (18.48) & (15.40) & (3.96) & (100) \\
\hline
\end{tabular}
\end{center}
\vspace*{-3ex}
\end{table}

Imagine now that the full population consisted of 50 million people, and
that the values of the two variables for them were as shown in Table
\ref{t_sex_attitude_pop}. The frequencies in this table desribe the
population distribution of the variables in this hypothetical
population, with the joint distribution of sex and attitude shown by the
internal cells of the table and the marginal distributions by its
margins. So there are for example 3.84 million men and 4.16 million
women in the population who strongly agree with the attitude statement,
and 1.98 million people overall who strongly
disagree with it.

Rather than the frequencies, it is more helpful to discuss population
distributions in terms of proportions. Table \ref{t_sex_attitude_pop}
shows two sets of them, the overall proportions [in square brackets] out
of the total population size, and the two rows of conditional
proportions of attitude given sex (in parentheses). Either of these can
be used to introduce the ideas of population distributions, but we focus
on the conditional proportions because they will  be more convenient for
the discussion in later chapters. In this population we observe, for
example, that the conditional proportion of ``Strongly disagree'' given
that a person is a woman is 0.03, i.e.\ 3\% of women strongly
disagree with the statement, while among men the corresponding
conditional proportion is 0.05.

Instead of ``proportions'', when we discuss population distributions we
will usually talk of ``probabilities''. The two terms are equivalent
when the population is finite and the variables are categorical, as in
Table \ref{t_sex_attitude_pop}, but the language of probabilities is
more appropriate in other cases. We can then say that Table
\ref{t_sex_attitude_pop} shows two sets of \textbf{conditional
probabilities} in the population, which define two conditional
\textbf{probability distributions} for attitude given sex.

The notion of a probability distribution creates a conceptual connection
between population distributions and sampling from them. This is that
the probabilities of the population distribution can also be thought of
as sampling probabilities in (simple random) sampling from the
population. For example, here the conditional probability of ``Strongly
disagree'' among men is 0.05, while the probability of ``Strongly
agree'' is 0.16. The sampling interpretation of this is that if we
sample a man at random from the population, the probability is 0.05
that he strongly disagrees and 0.16 that he strongly agrees with the
attitude statement.

The view of population distributions as probability distributions works
also in other cases than the kind that is illustrated by Table
\ref{t_sex_attitude_pop}. First, it applies also for continuous
variables, where proportions of individual values are less useful (this
is discussed further in Chapter \ref{c_means}). Second, it is also
appropriate when the population is regarded as an infinite
superpopulation, in which case the idea of population \emph{frequencies}
is not meaningful. With this device we have thus reached a formulation of
a population distribution which is flexible enough to cover all the
situations where we will need it.

\section{Need for statistical inference}
\label{s_samples_inference}

We have now introduced the first key concepts that are involved in statistical
inference:
\begin{itemize}
\item
The population, which may regarded as finite or infinite. Distributions
of variables in the population are the population distributions, which
are formulated as probability distributions of the possible values of
the variables.
\item
Random samples from the population, and sample distributions of
variables in the sample.
\end{itemize}
Substantive research questions are most often questions about population
distributions. This raises the fundamental challenge of inference: what
we are interested in --- the population --- is not fully observed, while
what we do observe --- the sample --- is not of main interest for
itself. The sample is, however, what information we do have to draw on
for conclusions about the population. Here a second challenge arises:
because of random variation in the sampling, sample distributions will
not be identical to population distributions, so inference will not be
as simple as concluding that whatever is true of the sample is also true
of the population. Something cleverer is needed to weigh the evidence in
the sample, and that something is statistical inference.

The next three chapters are mostly about statistical inference. Each of
them discusses a particular type of analysis and inferential and
decriptive statistical methods for it. These methods are some of the
most commonly used in basic statistical analyses of empirical data. In
addition, we will also use them as contexts in which to introduce the
general concepts of statistical inference. This will be done gradually,
with each chapter both building on previous concepts and introducing new
ones, as follows:
\begin{itemize}
\item
Chapter \ref{c_tables}: Associations in two-way contingency tables
(significance testing, sampling distributions of statistics).
\item
Chapter \ref{c_probs}: Single proportions and comparisons of
proportions (probability distributions, parameters, point estimation, confidence
intervals).
\item
Chapter \ref{c_means}: Means of continuous variables (probability distributions
of continuous variables, and inference for such variables).
\end{itemize}




